\documentclass[12pt,table,xcolor={dvipsnames}]{beamer}
\usetheme{Pittsburgh}
\usecolortheme{seagull}
%\usepackage[utf8]{inputenc}
\usepackage{fontspec}
\usepackage{amsmath}
\usepackage{listings}
\usepackage{multirow}
\usepackage{amsfonts}
\usepackage{amssymb}
\usepackage{graphicx}
\author{Design and Verification of Security Protocols and Security Ceremonies}
\title{\vspace{-.2cm}Protocol Verification Techniques - Theorem Provers}
%\setbeamercovered{transparent} 
\setbeamertemplate{navigation symbols}{} 
%\logo{\includegraphics[scale=0.015]{Brasao_UFSC.png}\includegraphics[scale=0.2]{brasao_PPGCC.jpg}} 
\institute{Programa de Pós-Graduacão em Ciências da Computacão \\ Dr. Jean Everson Martina} 
\date{\vspace{-1cm}March-June 2018} 
\subject{} 
\usebackgroundtemplate{\includegraphics[width=\paperwidth,
height=\paperheight]{../reusable_images/fundo_UFSC.png}}
\begin{document}

{
\usebackgroundtemplate{\includegraphics[width=\paperwidth,
height=\paperheight]{../reusable_images/fundo_capa.png}}
\begin{frame}
\titlepage
\includegraphics[scale=0.3]{../reusable_images/brasao_PPGCC.jpg}
\end{frame}
}

\frame{
	\frametitle{Attention!}
\begin{block}{Attention!}
This topic will be divided into two lectures. One will deal with automatic theorem provers using FOL and the second will deal with theorem provers using HOL
\end{block}
}

\frame{
	\frametitle{Security Protocol Analysis using Theorem Proving}

}

\frame{
\frametitle{A Small Review on Logics}
Before we get our hand on theorem proving we need to talk a bit about logics:\pause
\begin{itemize}
 \item Propositional Logic
 \item First-Order Logic (FOL)
\end{itemize}
}

\frame{
\frametitle{Propositional Logic}
\begin{itemize}
\item It is basic logical system;\pause
\item It studies its arguments and their structure;\pause
	\begin{itemize}
	\item An argument is a declarative sentence in natural language like English;\pause
	\item Example: \textit{``The bus is late"} \pause
	\end{itemize}
 \item It was discovered by Aristóteles in ancient Greece;\pause
\item Each sentence receive a truth value being $T$ (True) or $F$ (False).
\end{itemize}
}

\frame{
\frametitle{Propositional Logic}
\begin{itemize}
\item There are well defined rules to extract meaning from complex arguments:\pause
\begin{itemize}
\item Modus Ponens;\pause
\item Modus Tolens;\pause
\item Negation of the implication, etc;\pause
\end{itemize}
\item It is a classical logic that is easy to understand.
\end{itemize}
}

\frame{ 
\frametitle{An Example Formula in Propositional Logic}
\begin{center}
\begin{tabular}{ l  c  l }
\textit{It is Raining.} &:& $P$ \\\pause
\textit{Jane carries her umbrella.} &:& $Q$ \\\pause
\textit{Jane gets wet.} &:& $R$\pause
\end{tabular}
\end{center}

\begin{center}
$(P \land \lnot Q) \to R, \lnot R, P \vdash Q$\pause
\end{center}

\begin{itemize}
\item \textbf{Symbols}: $\land$ -- ``and"; $\lor$ -- ``or"; $\not$ -- ``not"; $\to$ -- ``implies"; $\leftrightarrow$ -- ``equivalent"; $\vdash$ -- ``proves"; and $\not \vdash$ -- ``do not prove".
\end{itemize}
}

\frame{
\frametitle{First-Order Logic (FOL)}
\begin{itemize}
 \item It is also known as \textit{Predicate's Logic} ou \textit{Quantificational Logic};\pause
\item Extends the expressiveness of propositional logic;\pause
	\begin{itemize}
	 \item It is hard to express sentence like \textit{``Somethins  \textbf{is} or \textbf{has} ..."} in propositional logic;\pause 
	\end{itemize}
 \item The main difference of First-Order Logic is the existence of quantifiers:\pause
	\begin{itemize}
	 \item $\exists$ (there exists), and $\forall$ (for all);\pause
	\end{itemize}
\item Other concepts are: predicates, variables, functions and constants;\pause
\item This logic is expressive enough to verify security protocols.
\end{itemize}
}

\frame{
\frametitle{An Example Formula in FOL}
\begin{center}
\begin{tabular}{ l  c  l }
$S(x, y)$ &:& $x$ is Son of $y$ ($S$ is a predicate) \\\pause
$B(x,y)$ &:& $x$ is Brother of $y$ ($B$ is another predicate) \\\pause
$f(x)$ &:& returns the father of $x$ ($f$ is a function)\pause
\end{tabular}
\end{center}

\begin{center}
$\forall x [ S(x, f(m) \to B(x, m) ) ]$\newline ($m$ is a constant, and $x$ is a variable)\pause
\end{center}
Our protocols will be modelled this way!
}


\frame{
\frametitle{Defining Predicates for Protocol Verification}
\begin{itemize}
	\item $E(x)$: $x$ is an entity (agent) in the protocols;\pause
	\item $Stores(x,y)$: $x$ is stored by entity $y$;\pause
	\item $Knows(x,y)$: $x$ is known by entity  $y$;\pause
	\item $M(x)$: a message $x$ is sent in the protocol.
\end{itemize}
}

\frame{
\frametitle{Defining Functions for Protocol Verification}
	\begin{itemize}
	\item Grouping message components:
		\begin{itemize}
		\item  $pair(x,y)$; $triple(x,y,z)$;\pause
		\end{itemize}


	\item Message exchange: 
		\begin{itemize}
		\item $sent(x,y,z)$: agent $x$ sends to agent $y$ message $z$;\pause
		\end{itemize}

	\item Key related functions:
		\begin{itemize}
		\item $krkey(x,y)$: private key $x$ belongs to agent $y$;\pause
		\item $kukey(x,y)$:  $x$ belongs to agent $y$; and \pause
		\item $kp(x,y)$: private key $x$ and public key $y$ make a key pair.
		\end{itemize}
	\end{itemize}
}

\frame{
\frametitle{Defining Functions for Protocol Verification}

\begin{itemize}
	\item \textit{Nonce} functions:
		\begin{itemize}
		\item $nonce(x,y)$: \textit{nonce} $x$ is generated by entity $y$;\pause
		\end{itemize}
	\item Cryptographic primitives:
		\begin{itemize}
		\item $encr(x,y)$: $x$ is encrypted using key $y$; and\pause
		\item $sign(x,y)$: $x$ is signed using key $y$.
		\end{itemize}
\end{itemize}
}

\frame{
\frametitle{Defining Constants for Protocol Verification}
	\begin{itemize}
	\item Agents participating in the protocols:
		\begin{itemize}
		\item $a$ (Alice); $b$ (Bob); $c$ (Charlie).\pause
		\end{itemize}
	\item Private keys and public keys:
		\begin{itemize}
		\item $kra$; $kua$: private key and public key belonging to Alice;
		\item $krb$; $kub$: private key and public key belonging to Bob;
		\item $krc$; $kuc$: private key and public key belonging to Charlie;\pause
		\end{itemize}
	\item \textit{Nonces}: 
		\begin{itemize}
		\item $na$; $nb$; $nc$.
		\end{itemize}
\end{itemize}
}

\frame{\frametitle{Needham-Schroeder Public Key Protocol}

		\begin{table}[htdp]
\begin{center}
\begin{tabular}{ l l }
1. & A $\rightarrow$ B:  $\{| N_a , A |\}_{K_b}$ \\ 
2. & B $\rightarrow$ A:  $\{| N_a , N_b |\}_{K_a}$ \\ 
3. & A $\rightarrow$ B:  $\{| N_b |\}_{K_b}$ \\
\end{tabular}
\end{center}
\end{table}%
}

\begin{frame}{NSPKP Goals}
\begin{itemize}
\item The goal of the protocol is to establish mutual authentication between two parties A and B in the presence of adversary;
\item A and B obtain a secret shared key though direct communication using public key cryptography;
\item This adversary can intercept messages, delay messages, read and copy messages and generate messages;
\item This adversary can not learn the privates keys of principals.
\end{itemize}
\end{frame}

\frame{
\frametitle{Defining Initial Knowledge}
\begin{itemize}
\item A first step is to define the initial knowledge of each agent;\pause
\item For example Alice's is:\pause
\end{itemize}
\begin{exampleblock}{Example}
$E(a)$\label{form:admin-is-entity}\newline
$Knows( kp( krkey( kra, a ), kukey( kua, a ) ), a )$\label{form:admin-knows-A-keypair}\newline
$Knows( kukey( kub, b ), a)$\label{form:admin-knows-C-pubkey}\newline
$Knows( kukey( kuc, c ), a)$\label{form:admin-knows-C-pubkey}\newline
$Knows( nonce( na, a ), a )$\label{form:admin-knows-na}
\end{exampleblock}\pause
We do the same thing to Bob and Charlie changing the constants only.
}

\frame{
\frametitle{Describing NSPKP}

\begin{itemize}
\item We model the message exchange;\pause
\item The first massage is modelled to:\pause
\end{itemize}
\begin{exampleblock}{Example}
\small{
$Knows( kukey( kua, a ), a ) \wedge \newline Knows( kp( krkey( kra, a ), kukey( kua, a ) ), a )  \wedge \newline Knows( kukey( kub, b ), a) \wedge \newline Knows( nonce( na, a ), a ) \newline \to\newline M( sent( a, b, encr( pair( na, a ), kub ) ) \wedge \newline Stores( pair(na, b) a )$\label{form:reg:1}
}
\end{exampleblock}
}

\frame{
\frametitle{Describing NSPKP}
\begin{itemize}
\item The second massage is modelled to:\pause
\end{itemize}
\begin{exampleblock}{Example}
\small{
$\forall x [Knows( kukey( kub, b ), b ) \wedge \newline Knows( kp( krkey( krb, b ), kukey( kub, b ) ), b )  \wedge \newline Knows( kukey( kua, a ), b) \wedge \newline Knows( nonce( nb, b ), b ) \wedge \newline M( sent( x, b, encr( pair( na, a ), kub ) )  \newline \to \newline M( sent( b, a, encr( pair( na, nb ), kua ) ) \wedge \newline Stores( pair(nb, a), b )  ] $\label{form:reg:6}
}
\end{exampleblock}
}

\frame{
\frametitle{Describing NSPKP}
\begin{itemize}
\item The third massage is modelled to:\pause
\end{itemize}
\begin{exampleblock}{Example}
\small{
$\forall x [\newline Stores( pair(na,b), a )  \wedge \newline M( sent( x, a, encr( pair( na, nb ), kua ) )  \newline \to \newline M( sent( a, b, encr( nb ), kub ) ) ] $\label{form:reg:6}
}
\end{exampleblock}
}


\frame{
\frametitle{Logical Model for the Attacker}
\begin{itemize}
\item The attacker is a Dolev-Yao one;\pause
\item The attacker model add some logical elements:\pause
	\begin{itemize}
	\item The constant $c$ which represents the attacker himself;\pause
	\item The attacker data when impersonating another valid user int he protocol;\pause
	\item The predicate $Im(x)$ which holds the knowledge acquired by the attacker by the manipulation of the exchange messages;\pause
	\item This predicate work is a similar way to $M(x)$ predicate.
	\end{itemize}
\end{itemize}
}

\frame{
\frametitle{Attacker's Initial Knowledge}
\begin{enumerate}

\item The attacker is an entity of the protocols and has its own lawful data set:
	\begin{itemize}
	\item $E(c)$\label{form:Intruder-is-entity}\pause
	\item He also has his key pair and nonce which are omitted here;\pause
	\end{itemize}

\item He knows the public data of other agents:
	\begin{itemize}
	\item $Knows( kukey( kua, a ), c )$\label{form:Intruder-knows-C-pubkey}
	\item $Knows( kukey( kub, b ), c )$\label{form:Intruder-knows-A-pubkey}\pause
	\end{itemize}

\item He can record all the messages:
	\begin{itemize}
	\item $\forall x, y, w [ M(sent(x,y,w))\to Im(w) ]$\label{form:Intruder-records-all}
	\end{itemize}
\end{enumerate}
}



\frame{
\frametitle{Attacker's Messages Manipulation Capabilities}
\begin{enumerate}
\item He can decompose the message into smaller pieces:
	\begin{itemize}
	\item $\forall u, v [ Im( pair( u, v ) ) \to Im( u ) \wedge Im( v ) ] $\label{form:Intruder-decomposes-pairs}
	\item $\forall u, v, w [ Im( triple( u, v, w ) ) \to Im( u ) \wedge Im( v ) \wedge Im( w ) ]$\label{form:Intruder-decomposes-triples}\pause
	\end{itemize}

\item He can fabricate message form the knowledge he acquired:
	\begin{itemize}
	\item $\forall u, v  [ Im( u ) \wedge Im( v ) \to Im( pair( u, v ) ) ] $\label{form:Intruder-assembles-pairs}
	\item $\forall u, v, w  [ Im( u ) \wedge Im( v ) \wedge Im( w ) \to Im( triple( u, v, w ) ) ] $\label{form:Intruder-assembles-triples}\pause
	\end{itemize}

\item He can send fake messages:
	\begin{itemize}
	\item $\forall u, x, y [ Im( u ) \wedge E( x ) \wedge E( y ) \to M( sent( x, y, u ) ) ] $\label{form:Intruder-sends-fake-messages}
	\end{itemize}
\end{enumerate}
}


\frame{
\frametitle{Attacker's Cryptographic Capabilities}
\begin{enumerate}
% Anything can potentially be a key
\item Anything can potentially be a key:
	\begin{itemize}
	\item $\forall u, v [  Im( u ) \wedge E( v ) \to Knows( krkey( u, v ), c ) ] $\label{form:Intruder-tests-for-privkeys}
	\item $\forall u, v [  Im( u ) \wedge E( v ) \to Knows( kukey( u, v ), c ) ] $\label{form:Intruder-tests-for-pubkeys}\pause
	\end{itemize}
%Anything can potentially be a nonce
\item Anything can potentially be a \textit{nonce}:
	\begin{itemize}
	\item $\forall u, v [  Im( u ) \wedge E( v ) \to Knows( nonce( u, v ), c ) ] $\label{form:Intruder-tests-for-nonces}\pause
	\end{itemize}
% Generate encrypted messages
% Generate signed messages
\item He can encrypt and sing with known keys:
	\begin{itemize}
	\item $\forall u, v, x [  Im( u ) \wedge Knows( kukey( v, x ), c ) \wedge E( x ) \to Im( encr( u, v ) ) ] $\label{form:Intruder-generates-encrypted-messages}
	\item $\forall u, v, x [ Im( u ) \wedge Knows( krkey( v, x ), c ) \wedge E( x ) \to Im( sign( u, v ) ) ] $\label{form:Intruder-generates-signed-messages}
	\end{itemize}
\end{enumerate}
}


\frame{
\frametitle{More Attacker's Cryptographic Capabilities}
\begin{enumerate}
% Decrypts with known keys
\item Decrypt messages with known keys:
	\begin{itemize}
	\item $\forall u, v, w, x [ Im( encr( u, v ) ) \wedge Knows( kp( krkey( w, x ), kukey( v, x ) ), c ) \wedge \newline E( x ) \to Im( u ) ) ] $\label{form:Intruder-decrypts-known-keys}
	\end{itemize}\pause

% Decrypts with known nonces
\item Decrypt messages with known \textit{nonces}:
	\begin{itemize}
	\item $\forall u, v, w [ Im( encr( u, v ) ) \wedge Knows( nonce( v, w ), c ) \wedge E( w ) \to Im( u ) ) ] $\label{form:Intruder-decrypts-known-nonces}
	\end{itemize}\pause

% Access contents of signed messages
\item Learn signed messages:
	\begin{itemize}
	\item $\forall u, v [ Im( sign( u, v ) ) \to Im( u ) ]$\label{form:Intruder-access-signed-content}
	\end{itemize}
\end{enumerate}
}


\frame{
\frametitle{All that can be Mechanised}
\begin{itemize}
\item The proofs can be made manually with pen and paper;\pause
\item However it is more convenient to use theorem prover to do the hard work;\pause
\item Any FOL capable theorem prober can to the job;\pause
\item We like a lot SPASS:
	\begin{itemize}
	\item Deals with FOL;
	\item Makes proofs by contradiction;\pause
	\item General use.
	\end{itemize}
\end{itemize}
}

\frame{
\frametitle{Testing the Logical Model}
\begin{itemize}
\item We need to write conjectures;\pause
\item the theorem prover will tell us if it is true or not:\pause
	\begin{itemize}
	\item Conjectures are statements that we don know if they are true or not from the axioms;\pause
	\item We can then extract knowledge from the test of conjectures;\pause
	\end{itemize}
\item Lowe's attack can be easily reproduced with this setting we just saw;
\end{itemize}
}

\frame{
\frametitle{Problems with Protocol Verification with FOL}
\begin{itemize}
\item By definition FOL is non decidable;\pause
\item We need to use a subset that are the Monadic Horn Clauses;\pause
\item The proofs are as good as the conjectures created;\pause
\item If you are not clever it will not work.
\end{itemize}
}

 
\frame{
	\frametitle{Discussion}
\begin{itemize}
\item What else can you foresee modelled using this strategy?\pause
\item Can this be extended?\pause
\item What this strategy can not do?
\end{itemize}
}

{
\usebackgroundtemplate{\includegraphics[width=\paperwidth,
height=\paperheight]{../reusable_images/fundo_capa.png}}
\begin{frame}

{\LARGE Questions????}

\end{frame}
}

{
\usebackgroundtemplate{\includegraphics[width=\paperwidth,
height=\paperheight]{../reusable_images/fundo_capa.png}}
\begin{frame}
\includegraphics[scale=0.8]{../reusable_images/cc_logo_arge.png}\hspace{0.5cm} 
\includegraphics[scale=0.95]{../reusable_images/by.png}

\vspace{1cm}
This work is licensed under the Creative Commons Attribution 4.0 International License. To view a copy of this license, visit http://creativecommons.org/licenses/by/4.0/.
\end{frame}
}

\end{document}