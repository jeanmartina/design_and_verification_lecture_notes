\documentclass[12pt,table,xcolor={dvipsnames}]{beamer}
\usetheme{Pittsburgh}
\usecolortheme{seagull}
%\usepackage[utf8]{inputenc}
\usepackage{fontspec}
\usepackage{amsmath}
\usepackage{listings}
\usepackage{multirow}
\usepackage{amsfonts}
\usepackage{amssymb}
\usepackage{graphicx}
\author{Design and Verification of Security Protocols and Security Ceremonies}
\title{\vspace{-1.2cm}Security Properties, Advanced Security\\\vspace{1.2cm}Properties and Properties Composition}
%\setbeamercovered{transparent} 
\setbeamertemplate{navigation symbols}{} 
%\logo{\includegraphics[scale=0.015]{Brasao_UFSC.png}\includegraphics[scale=0.2]{brasao_PPGCC.jpg}} 
\institute{Programa de Pós-Graduacão em Ciências da Computacão \\ Dr. Jean Everson Martina} 
\date{\vspace{.2cm}March-June 2019}  
\subject{} 
\usebackgroundtemplate{\includegraphics[width=\paperwidth,
height=\paperheight]{../reusable_images/fundo_UFSC.png}}
\begin{document}

{
\usebackgroundtemplate{\includegraphics[width=\paperwidth,
height=\paperheight]{../reusable_images/fundo_capa.png}}
\begin{frame}
\titlepage
\includegraphics[scale=0.3]{../reusable_images/brasao_PPGCC.jpg}
\end{frame}
}

\begin{frame}{Security Protocols}
\begin{itemize}
\item The focus of security protocols is on secure communications;\pause
\item Two or more parties are involved;\pause
\item Communication is carried over an insecure network;\pause
\item Cryptography is used to achieve some goal.
\end{itemize}
\end{frame}

\begin{frame}{Security Protocols Goals}
\begin{itemize}
\item Guided by user needs:\pause
\begin{itemize}
\item Provide an end-to-end encryption channel;\pause
\item Authenticate peers;\pause
\item Enable secure money transfer;\pause
\item Provide anonymity;\pause
\item Authenticate data;\pause
\end{itemize}
\item Usually are a claim of designers that must be verified.
\end{itemize}
\end{frame}

\begin{frame}{Building Blocks}
\begin{itemize}
\item Symmetric cryptography;\pause
\item Cryptographic hashes;\pause
\item Asymmetric cryptography;\pause
\item Advanced primitives;\pause
\item Other security protocols;
\end{itemize}
\end{frame}

\begin{frame}{Standard Security Properties}
\begin{itemize}
\item Confidentiality;\pause
\item Integrity;\pause
\item Timeliness;\pause
\item Authentication.
\end{itemize}
\end{frame}

\begin{frame}{Advanced Security Properties}
\begin{itemize}
\item Forward secrecy;\pause
\item Non-repudiation;\pause
\item Availability;\pause
\item Anonymity rather than Authenticity;\pause
\item Plausible Deniability rather than Non-Repudiation;\pause
\item Transparency instead of Privacy; \pause
\item Etc...
\end{itemize}
\end{frame}

\begin{frame}{Confidentiality}
\begin{itemize}
\item Also called secrecy;\pause
\item Is the usual main goal of cryptography;\pause
\item Can be provided using symmetric cryptography or asymmetric cryptography;\pause
\item Symmetric cryptography is not ``clean cut" since it always provide some sort of authentication; \pause
\item Asymmetric cryptography separate confidentiality from authentication.
\end{itemize}
\end{frame}

\begin{frame}{Confidentiality Examples}
\begin{itemize}
\item A sends to B message M encrypted with shared key Kab;\pause
\item A sends to B message M encrypted with B's public key.
\end{itemize}
\end{frame}

\begin{frame}{Integrity}
\begin{itemize}
\item Is the main property provided by hash functions and message authentication codes;\pause
\item Hash function provide ``clean cut'' integrity checking;\pause
\item MACs provide integrity coupled with authentication;\pause
\item Integrity provided by theses cryptographic primitives are intended to detect active modification of messages;\pause
\item Integrity functions can also be used to avoid homomorphic manipulation;\pause
\item Can be used to avoid reverting operations within protocols;\pause
\item On itself is a weak property.
\end{itemize}
\end{frame}

\begin{frame}{Integrity Examples}
\begin{itemize}
\item A sends to B the hash of message M;\pause
\item A sends to B the authentication code of message M with Key Kab.
\end{itemize}
\end{frame}

\begin{frame}{Timeliness}
\begin{itemize}
\item Anchor the messages to the correct timing;\pause
\item Can be provided by nonces (number used only once);\pause
\item Can be provided by Timestamps;\pause
\item Timestamps are usually coupled with time to live requirements;\pause
\item Allows for peers to check the ordering of messages;\pause
\item Allows for peers to check the liveness of other peers.
\end{itemize}
\end{frame}

\begin{frame}{Timeliness Examples}
\begin{itemize}
\item A sends to B the nounce Na and recieves back Nb,Na ;\pause
\item A sends to B the timestamp of the generation time of the messages.
\end{itemize}
\end{frame}

\begin{frame}{Authentication}
\begin{itemize}
\item Is a basic but usually composed property;\pause
\item Comes in different shapes depending on the basic building blocks used;\pause
\item Aliveness - A runs the protocol with B;\pause
\item Weak Agreement - A runs the protocol with B but B does not authenticate A;\pause
\item Non-Injective Agreement - Key exchange;\pause
\item Mutual Agreement - A runs the protocol with B but B does authenticate A.
\end{itemize}
\end{frame}

\begin{frame}{Two facets of authentication}
\begin{itemize}
\item Authentication can serve both for assigning responsibility and for giving credit;\pause
\item An ``authenticated'' message M from a principal A to a principal B may be used in at least two distinct ways:\pause
\begin{itemize}
\item B may believe that the message M is being supported by A’s authority;\pause
\item B may attribute credit for the message M to A.\pause
\end{itemize}
\end{itemize}
\end{frame}


\begin{frame}{Two facets of authentication}
\begin{itemize}
\item Some protocols are adequate for assigning responsibility but not for giving credit, and vice versa;\pause
\item The two facets of authentication are most clearly separate in protocols
that rely on asymmetric cryptosystems;\pause
\item Even when it is proved beyond a reasonable doubt that a principal sent a message, responsibility and credit may not follow.
\end{itemize}
\end{frame}

\begin{frame}{Views on responsibility and credit}
\begin{itemize}
\item An authentication protocol should at least establish responsibility;\pause
\item There does not seem to be a consensus that an authentication protocol should also
establish credit;\pause
\item Once a protocol has set up a channel that speaks for a principal, it is easy to use the channel for establishing credit whenever the need arises;\pause
\item Establishing credit is a matter of prudence.
\end{itemize}
\end{frame}

\begin{frame}{Analysis of Authentication}
\begin{itemize}
\item Honest protocol participants are expected to follow the rules of the protocol faithfully,and not to try to obtain credit for messages that they did not generate
themselves. A proof about honest protocol participants may show that a protocol
establishes responsibility, but not credit;\pause
\item When an attacker is included as protocol participant, the attacker is not forced
to follow the rules of the protocol, and may attempt to get undue credit. A proof
that concerns such an attacker can show that a protocol establishes credit.
\end{itemize}
\end{frame}

\begin{frame}{List of Advanced Security Properties}
\begin{itemize}
\item Forward secrecy;\pause
\item Non-repudiation;\pause
\item Plausible Deniability;\pause
\item Availability;\pause
\item Eligibility;\pause
\item Fairness;\pause
\item Receipt-freeness;\pause
\item Coercion-resistance;\pause
\item Privacy; \pause
\item Anonymity;\pause
\item Transparency.
\end{itemize}
\end{frame}


\begin{frame}{Forward Secrecy}
\begin{itemize}
\item Forward secrecy relates to the non-interference of short term keys leakage to new short term keys;\pause
\item It does not relate to long term keys;\pause
\item Is usually obtained by the negotiation of short term keys only based on long term keys;\pause
\item Has a complementary property that is Backwards secrecy;\pause
\item Having both lead to full non-interference among session keys.
\end{itemize}
\end{frame}

\begin{frame}{Non-repudiation}
\begin{itemize}
\item Is the inability to deny knowledge of a message;\pause
\item Happens as non-repudiation of origin, meaning authorship;\pause
\item Happens as non-repudiation of destiny, meaning confirmation of reception;\pause
\item Is usually implemented using asymmetric crypto in digital signature mode;\pause
\item Can also be achieve by the use of commitments.
\end{itemize}
\end{frame}

\begin{frame}{Plausible Deniability}
\begin{itemize}
\item Is the ability to deny knowledge of a message;\pause
\item Act the the courter property of Non-Repudiation;\pause
\item Is implemented shared secret crypto;\pause
\item Can also happen on origin, destination or both.
\end{itemize}
\end{frame}

\begin{frame}{Availability}
\begin{itemize}
\item Is the property that relates to presence of knowledge whenever needed;\pause
\item Is difficult to reach by crypto-means;\pause
\item Is usually reached using replication;\pause
\item There are some interesting primitives that achieve availability such as secret-sharing.
\end{itemize}
\end{frame}

\begin{frame}{Eligibility}
\begin{itemize}
\item Is the property that states authority to a peer to act;\pause
\item Usually present on election protocols;\pause
\item Is related and derived from Authentication;\pause
\item Can also happen through delegation;\pause
\item Is implemented in this later case by the usage of tickets;\pause
\item Can also control the number of times the peer is allowed to do something.
\end{itemize}
\end{frame}

\begin{frame}{Fairness}
\begin{itemize}
\item Fairness is the properties that guarantees that no information is acquired out of the right time;\pause 
\item In election protocols it means that no early results can be obtained which could influence the remaining voters;\pause
\item Is usually implemented with encryption (either symmetric or asymmetric);\pause
\item The keys are them distributed is such a way that only an agreement can enable decryption.
\end{itemize}
\end{frame}

\begin{frame}{Receipt-freeness}
\begin{itemize}
\item Receipt-freeness is the property that the peer does not carry any proof of acts within the protocol;\pause
\item In election protocols it means that a voter does not gain any information (a receipt) which can be used to prove to a coercer that she voted in a certain way;\pause
\item It is tricky to achieve when combined with other properties;\pause
\item Implementation usually is not done using cryptographic means.
\end{itemize}
\end{frame}

\begin{frame}{Coercion-resistance}
\begin{itemize}
\item Coercion-resistance is the property that avoid a peer to act in certain way against its own will and forced by an external entity;\pause
\item In election protocols it means that a voter cannot cooperate with a coercer to prove to him that she voted in a certain way;\pause
\item It is usually achieve by using the last commitment within the protocol;\pause
\item Implementation usually depends of Receipt-freeness but is not a requirement.
\end{itemize}
\end{frame}

\begin{frame}{Verifiability}
\begin{itemize}
\item Is the property that allows for peers to be assured that their interaction was perceived within the protocol;\pause
\item Is usually implemented using bulletin boards;\pause
\item In election protocols it can be specialised in:\pause
\begin{itemize}
\item Individual verifiability: a voter can verify that her vote was really counted;\pause
\item Universal verifiability: the published outcome really is the sum of all the votes.
\end{itemize}
\end{itemize}
\end{frame}

\begin{frame}{Privacy}
\begin{itemize}
\item Privacy is the property that allows for peers to choose the amount of data that is being release to other peers;\pause
\item Has a controversial definition since it is related to a personal feeling;\pause
\item It is intrinsically related to Confidentiality;\pause
\item In election protocols it means that the system cannot reveal how a particular voter voted;\pause
\end{itemize}
\end{frame}

\begin{frame}{Anonymity}
\begin{itemize}
\item Anonymity is the property that does not allow identification of peers;\pause
\item Is usually achieve by using obfuscation techniques;\pause
\item Usually is implemented with hashes or MACs;\pause
\item Is a form of plausible deniability.
\end{itemize}
\end{frame}

\begin{frame}{Transparency}
\begin{itemize}
\item Transparency can be defined the distance of a message from ground truth;\pause
\item It a very new property that is actually being studied still;\pause
\item Is present in crypto-currency protocols and in health related systems;\pause
\item Is a dichotomy of Privacy.
\end{itemize}
\end{frame}

\begin{frame}{Discussion}
\begin{itemize}
\item Which other properties did you hear about?\pause
\item Which are the dichotomies you can see between the properties shown today?\pause
\item Can you foresee an online activity that you require a property not listed here?
\end{itemize}
\end{frame}


\begin{frame}{Discussion}
\begin{itemize}
\item Can give examples of security protocols that have these properties we shown above?\pause
\item Can you give examples of problems/attacks on security protocols that have these properties we shown above?\pause
\item How can we avoid problems/attacks on security protocols?
\end{itemize}
\end{frame}

{
\usebackgroundtemplate{\includegraphics[width=\paperwidth,
height=\paperheight]{../reusable_images/fundo_capa.png}}
\begin{frame}

{\LARGE Questions????}

\end{frame}
}

{
\usebackgroundtemplate{\includegraphics[width=\paperwidth,
height=\paperheight]{../reusable_images/fundo_capa.png}}
\begin{frame}
\includegraphics[scale=0.8]{../reusable_images/cc_logo_arge.png}\hspace{0.5cm} 
\includegraphics[scale=0.95]{../reusable_images/by.png}

\vspace{1cm}
This work is licensed under the Creative Commons Attribution 4.0 International License. To view a copy of this license, visit http://creativecommons.org/licenses/by/4.0/.
\end{frame}
}

\end{document}
