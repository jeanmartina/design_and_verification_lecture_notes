\documentclass[12pt,table,xcolor={dvipsnames}]{beamer}
\usetheme{Pittsburgh}
\usecolortheme{seagull}
%\usepackage[utf8]{inputenc}
\usepackage{fontspec}
\usepackage{amsmath}
\usepackage{listings}
\usepackage{multirow}
\usepackage{amsfonts}
\usepackage{amssymb}
\usepackage{graphicx}
\author{Design and Verification of Security Protocols and Security Ceremonies}
\title{\vspace{-.7cm}Security Properties}
%\setbeamercovered{transparent} 
\setbeamertemplate{navigation symbols}{} 
%\logo{\includegraphics[scale=0.015]{Brasao_UFSC.png}\includegraphics[scale=0.2]{brasao_PPGCC.jpg}} 
\institute{Programa de Pós-Graduacão em Ciências da Computacão \\ Dr. Jean Everson Martina} 
\date{\vspace{-1cm}August-November 2016} 
\subject{} 
\usebackgroundtemplate{\includegraphics[width=\paperwidth,
height=\paperheight]{../reusable_images/fundo_UFSC.png}}
\begin{document}

{
\usebackgroundtemplate{\includegraphics[width=\paperwidth,
height=\paperheight]{../reusable_images/fundo_capa.png}}
\begin{frame}
\titlepage
\includegraphics[scale=0.3]{../reusable_images/brasao_PPGCC.jpg}
\end{frame}
}

\begin{frame}{Security Protocols}
\begin{itemize}
\item The focus of security protocols is on secure communications;\pause
\item Two or more parties are involved;\pause
\item Communication is carried over an insecure network;\pause
\item Cryptography is used to achieve some goal.
\end{itemize}
\end{frame}

\begin{frame}{Security Protocols Goals}
\begin{itemize}
\item Guided by user needs:\pause
\begin{itemize}
\item Provide an end-to-end encryption channel;\pause
\item Authenticate peers;\pause
\item Enable secure money transfer;\pause
\item Provide anonymity;\pause
\item Authenticate data;\pause
\end{itemize}
\item Usually are a claim of designers that must be verified.
\end{itemize}
\end{frame}

\begin{frame}{Building Blocks}
\begin{itemize}
\item Symmetric cryptography;\pause
\item Cryptographic hashes;\pause
\item Asymmetric cryptography;\pause
\item Advanced primitives;\pause
\item Other security protocols;
\end{itemize}
\end{frame}

\begin{frame}{Standard Security Properties}
\begin{itemize}
\item Confidentiality;\pause
\item Integrity;\pause
\item Timeliness;\pause
\item Authentication.
\end{itemize}
\end{frame}

\begin{frame}{Advanced Security Properties}
\begin{itemize}
\item Forward secrecy;\pause
\item Non-repudiation;\pause
\item Availability;\pause
\item Anonymity rather than Authenticity;\pause
\item Plausible Deniability rather than Non-Repudiation;\pause
\item Transparency instead of Privacy; \pause
\item Etc...
\end{itemize}
\end{frame}

\begin{frame}{Confidentiality}
\begin{itemize}
\item Also called secrecy;\pause
\item Is the usual main goal of cryptography;\pause
\item Can be provided using symmetric cryptography or asymmetric cryptography;\pause
\item Symmetric cryptography is not ``clean cut" since it always provide some sort of authentication; \pause
\item Asymmetric cryptography separate confidentiality from authentication.
\end{itemize}
\end{frame}

\begin{frame}{Confidentiality Examples}
\begin{itemize}
\item A sends to B message M encrypted with shared key Kab;\pause
\item A sends to B message M encrypted with B's public key.
\end{itemize}
\end{frame}

\begin{frame}{Integrity}
\begin{itemize}
\item Is the main property provided by hash functions and message authentication codes;\pause
\item Hash function provide ``clean cut'' integrity checking;\pause
\item MACs provide integrity coupled with authentication;\pause
\item Integrity provided by theses cryptographic primitives are intended to detect active modification of messages;\pause
\item Integrity functions can also be used to avoid homomorphic manipulation;\pause
\item Can be used to avoid reverting operations within protocols;\pause
\item On itself is a weak property.
\end{itemize}
\end{frame}

\begin{frame}{Integrity Examples}
\begin{itemize}
\item A sends to B the hash of message M;\pause
\item A sends to B the authentication code of message M with Key Kab.
\end{itemize}
\end{frame}

\begin{frame}{Timeliness}
\begin{itemize}
\item Anchor the messages to the correct timing;\pause
\item Can be provided by nonces (number used only once);\pause
\item Can be provided by Timestamps;\pause
\item Timestamps are usually coupled with time to live requirements;\pause
\item Allows for peers to check the ordering of messages;\pause
\item Allows for peers to check the liveness of other peers.
\end{itemize}
\end{frame}

\begin{frame}{Timeliness Examples}
\begin{itemize}
\item A sends to B the nounce Na and recieves back Nb,Na ;\pause
\item A sends to B the timestamp of the generation time of the messages.
\end{itemize}
\end{frame}

\begin{frame}{Authentication}
\begin{itemize}
\item Is a basic but usually composed property;\pause
\item Comes in different shapes depending on the basic building blocks used;\pause
\item Aliveness - A runs the protocol with B;\pause
\item Weak Agreement - A runs the protocol with B but B does not authenticate A;\pause
\item Non-Injective Agreement - Key exchange;\pause
\item Mutual Agreement - A runs the protocol with B but B does not authenticate A.
\end{itemize}
\end{frame}

\begin{frame}{Two facets of authentication}
\begin{itemize}
\item Authentication can serve both for assigning responsibility and for giving credit;\pause
\item An ``authenticated'' message M from a principal A to a principal B may be used in at least two distinct ways:\pause
\begin{itemize}
\item B may believe that the message M is being supported by A’s authority;\pause
\item B may attribute credit for the message M to A.\pause
\end{itemize}
\end{itemize}
\end{frame}


\begin{frame}{Two facets of authentication}
\begin{itemize}
\item Some protocols are adequate for assigning responsibility but not for giving credit, and vice versa;\pause
\item The two facets of authentication are most clearly separate in protocols
that rely on asymmetric cryptosystems;\pause
\item Even when it is proved beyond a reasonable doubt that a principal sent a message, responsibility and credit may not follow.
\end{itemize}
\end{frame}

\begin{frame}{Views on responsibility and credit}
\begin{itemize}
\item An authentication protocol should at least establish responsibility;\pause
\item There does not seem to be a consensus that an authentication protocol should also
establish credit;\pause
\item Once a protocol has set up a channel that speaks for a principal, it is easy to use the channel for establishing credit whenever the need arises;\pause
\item Establishing credit is a matter of prudence.
\end{itemize}
\end{frame}

\begin{frame}{Analysis of Authentication}
\begin{itemize}
\item Honest protocol participants are expected to follow the rules of the protocol faithfully,and not to try to obtain credit for messages that they did not generate
themselves. A proof about honest protocol participants may show that a protocol
establishes responsibility, but not credit;\pause
\item When an attacker is included as protocol participant, the attacker is not forced
to follow the rules of the protocol, and may attempt to get undue credit. A proof
that concerns such an attacker can show that a protocol establishes credit.
\end{itemize}
\end{frame}

\begin{frame}{Discussion}
\begin{itemize}
\item Can give examples of security protocols that have these properties we shown above?\pause
\item Can you give examples of problems/attacks on security protocols that have these properties we shown above?\pause
\item How can we avoid problems/attacks on security protocols?
\end{itemize}
\end{frame}

{
\usebackgroundtemplate{\includegraphics[width=\paperwidth,
height=\paperheight]{../reusable_images/fundo_capa.png}}
\begin{frame}

{\LARGE Questions????}

\end{frame}
}

{
\usebackgroundtemplate{\includegraphics[width=\paperwidth,
height=\paperheight]{../reusable_images/fundo_capa.png}}
\begin{frame}
\includegraphics[scale=0.8]{../reusable_images/cc_logo_arge.png}\hspace{0.5cm} 
\includegraphics[scale=0.95]{../reusable_images/by.png}

\vspace{1cm}
This work is licensed under the Creative Commons Attribution 4.0 International License. To view a copy of this license, visit http://creativecommons.org/licenses/by/4.0/.
\end{frame}
}

\end{document}