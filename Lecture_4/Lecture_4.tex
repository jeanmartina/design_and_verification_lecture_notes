\documentclass[12pt,table,xcolor={dvipsnames}]{beamer}
\usetheme{Pittsburgh}
\usecolortheme{seagull}
%\usepackage[utf8]{inputenc}
\usepackage{fontspec}
\usepackage{amsmath}
\usepackage{listings}
\usepackage{multirow}
\usepackage{amsfonts}
\usepackage{amssymb}
\usepackage{graphicx}
\author{Design and Verification of Security Protocols and Security Ceremonies}
\title{\vspace{-.2cm}Classical Protocols \\ Needham-Schroeder Protocol Family}
%\setbeamercovered{transparent} 
\setbeamertemplate{navigation symbols}{} 
%\logo{\includegraphics[scale=0.015]{Brasao_UFSC.png}\includegraphics[scale=0.2]{brasao_PPGCC.jpg}} 
\institute{Programa de Pós-Graduacão em Ciências da Computacão \\ Dr. Jean Everson Martina} 
\date{\vspace{.2cm}March-June 2019} 
\subject{} 
\usebackgroundtemplate{\includegraphics[width=\paperwidth,
height=\paperheight]{../reusable_images/fundo_UFSC.png}}
\begin{document}

{
\usebackgroundtemplate{\includegraphics[width=\paperwidth,
height=\paperheight]{../reusable_images/fundo_capa.png}}
\begin{frame}
\titlepage
\includegraphics[scale=0.3]{../reusable_images/brasao_PPGCC.jpg}
\end{frame}
}

\begin{frame}{Needham-Schroeder Shared Key Protocol}
\begin{itemize}
\item Many existing protocols are derived from the work of Needham and Schroeder (1978);\pause 
\item Kerberos authentication protocol suite is one of the main used protocols and is derived from NSSKP;\pause
\item NSSKP is a shared-key authentication protocol designed to generate and propagate a session key which is used for subsequent symmetrically encrypted communication;\pause
\item There is no public key infrastructure in place.
\end{itemize}
\end{frame}

\begin{frame}{NSSKP Goals}
\begin{itemize}
\item The goal of the protocol is to establish mutual authentication between two parties A and B in the presence of adversary		;\pause
\item  A and B obtain a secret shared key though authentication server S;\pause
\item The adversary can intercept messages, delay messages, read and copy messages and generate messages;\pause
\item The adversary can not learn the secret keys of principals, which they share with the authentication server S.
\end{itemize}
\end{frame}

\begin{frame}{Assumptions of NSSKP}
\begin{itemize}
\item There are three principals, two named A and B, desiring mutual authentication, and
S, a trusted key server;\pause
\item It is assumed that A and B already have secure symmetric communication with
S using keys $K_{AS}$ and $K_{BS}$ , respectively;\pause
\item It is assumed that the attacker can not be a legitimate party within the protocol.
\end{itemize}
\end{frame}

\begin{frame}{Freshness Concepts of NSSKP}
\begin{itemize}
\item NSSKP uses nonces which are randomly generated values included in messages and used only once;\pause
\item If a nonce is generated and sent by one agent in one step and returned by another in a later step, the generator knows that the message is fresh and not a replay from an earlier exchange;\pause
\item Note that a nonce is not anchored in time. The only assumption is that it has not been used in any earlier interchange, with high probability because it is random and not used twice.
\end{itemize}
\end{frame}

\begin{frame}{How NSSKP works}
\begin{itemize}
\item A makes contact with the authentication server S, sending identities A and B and nonce NA;\pause
\item S responds with a message encrypted with the key of A. The message contains session key KAB (to be used by A and B) and certificate encrypted with B’s key conveying the session key and A’s identity;\pause
\item A sends the certificate to B;\pause
\item B decrypts the certificates and sends his own nonce encrypted by the session key to A; (nonce handshake);\pause
\item A decrypts the last message and sends modified nonce back to B.
\end{itemize}
\end{frame}

\begin{frame}{How NSSKP works}
\begin{block}{Goal}
By the end of the message exchange both A and B share the secret key and both are assured in the presence of each other. 
\end{block}
\end{frame}

\begin{frame}{How to Understand Security Protocols}
\begin{itemize}
\item There are questions to ask for any step in any protocol or ceremony:\pause
\begin{itemize}
\item What is the sender trying to say with this message?\pause
\item What is the receiver entitled to believe after receiving the message?\pause
\item Can I use less resources to achieve the same goals?\pause
\item Isn't there anything that I did not catch?
\end{itemize}
\end{itemize}
\end{frame}

\begin{frame}{Notation For Protocol Description}
\begin{table}[htdp]
\begin{center}
\begin{tabular}{ l l }
$A, B$ and $S$ & Agent names (Alice, Bob and Steve) \\\pause
$N_A$ & Random number chosen by Alice (Nonce)\\\pause
$K_{AS}$ & Long term key shared between Alice and Steve\\\pause
$\{X\}_{K_{AS}}$ & Encrypted message using $K_{AS}$\\
\end{tabular}
\end{center}
\label{default}
\end{table}%
\end{frame}

\begin{frame}{NSSKP Message Exchange}
\begin{table}[htdp]
\begin{center}
\begin{tabular}{ l l }
1. & A $\rightarrow$ S:  $ A , B, N_{A}$ \\\pause
2. & S $\rightarrow$ A:  $\{N_{A}, B, K_{AB}, \{K_{AB}, A\}_{K_{BS}}\}_{K_{AS}}$ \\\pause
3. & A $\rightarrow$ B:  $\{K_{AB}, A\}_{K_{BS}}$ \\\pause
4. & B $\rightarrow$ A:  $\{ N_B \}_{K_{AB}}$ \\\pause
5. & A $\rightarrow$ B:  $\{ N_B - 1 \}_{K_{AB}}$ \\
\end{tabular}
\end{center}
\end{table}%
\end{frame}

\begin{frame}{NSSKP - What is being said!}
\begin{table}[htdp]
\begin{center}
\begin{tabular}{ l l }
1. & A $\rightarrow$ S:  $ A , B, N_{A}$ \\\pause
& {\color{red}Hi Steve, this is Alice, I want to talk to Bob, }\\
& {\color{red}that's the identifier of my request.}\\\pause
2. & S $\rightarrow$ A:  $\{N_{A}, B, K_{AB}, \{K_{AB}, A\}_{K_{BS}}\}_{K_{AS}}$ \\\pause
& {\color{red}Alice I am sending you a secret which shows}\\ 
& {\color{red}your identifier, Bob's identity and the key }\\
& {\color{red}for you to talk to him. Here is a ticket to send}\\
& {\color{red}Bob the key and relate it to your identity.}\\\pause
3. & A $\rightarrow$ B:  $\{K_{AB}, A\}_{K_{BS}}$ \\\pause
& {\color{red}Bob there is a ticket for you!}\\ 
\end{tabular}
\end{center}
\end{table}%
\end{frame}


\begin{frame}{NSSKP - What is being said!}
\begin{table}[htdp]
\begin{center}
\begin{tabular}{ l l }
4. & B $\rightarrow$ A:  $\{ N_B \}_{K_{AB}}$ \\\pause
& {\color{red}I want to challenge Alice to see if she }\\ 
& {\color{red}has the key on the ticket.}\\ \pause
5. & A $\rightarrow$ B:  $\{ N_B - 1 \}_{K_{AB}}$ \\\pause
& {\color{red}Challenge accepted. Take it back!}\\ 
\end{tabular}
\end{center}
\end{table}%
\end{frame}

\begin{frame}{NSSKP Knowledge Exchange}
\begin{table}[htdp]
\begin{center}
\begin{tabular}{ l l }
& {\color{olive}Alice Knows $N_A$ and $K_{AS}$}\\ \pause
1. & A $\rightarrow$ S:  $ A , B, N_{A}$ \\\pause
& {\color{olive}Steve Knows $N_A$, $K_{AB}$, $K_{BS}$ and $K_{AS}$}\\ \pause
2. & S $\rightarrow$ A:  $\{N_{A}, B, K_{AB}, \{K_{AB}, A\}_{K_{BS}}\}_{K_{AS}}$ \\\pause
& {\color{olive}Alice Knows $N_A$, $K_{AS}$ and $K_{AB}$}\\ \pause
3. & A $\rightarrow$ B:  $\{K_{AB}, A\}_{K_{BS}}$ \\\pause
& {\color{olive}Bob Knows $N_B$, $K_{BS}$ and $K_{AB}$}\\ \pause
4. & B $\rightarrow$ A:  $\{ N_B \}_{K_{AB}}$ \\\pause
& {\color{olive}Alice Knows $N_A$, $N_B$, $K_{AS}$ and $K_{AB}$}\\ \pause
5. & A $\rightarrow$ B:  $\{ N_B - 1 \}_{K_{AB}}$ \\
\end{tabular}
\end{center}
\end{table}%
\end{frame}

\begin{frame}{Informal Verification of Security Protocols}
\begin{itemize}
\item What to ask about a protocol or ceremony:\pause
\begin{itemize}
\item Are both authentication and secrecy assured?\pause
\item Is it possible to impersonate one or more of the parties?\pause
\item Is it possible to interject messages from an earlier exchange (replay attack)?\pause
\item What tools can an attacker deploy?\pause
\item If any key is compromised, what are the consequences?
\end{itemize}
\end{itemize}
\end{frame}

\begin{frame}{Denning and Sacco Attack}
\begin{itemize}
\item Denning and Sacco pointed out that the compromise of a session key has bad consequences. An intruder can reuse an old session key and pass it off as a new one;\pause
\item Suppose Charlie has acquired $K_{AB}$ from a past run of the protocol, and start the protocol from message 3;\pause
\end{itemize}
\begin{block}{Bob will believe he is talking to Alice}\pause
\begin{table}[htdp]
\begin{center}
\begin{tabular}{ l l }
3. & C $\rightarrow$ B:  $\{K_{AB}, A\}_{K_{BS}}$ \\\pause
4. & B $\rightarrow$ C:  $\{ N_B \}_{K_{AB}}$ \\\pause
5. & C $\rightarrow$ B:  $\{ N_B - 1 \}_{K_{AB}}$ \\
\end{tabular}
\end{center}
\end{table}%
\end{block}
\end{frame}

\begin{frame}{Denning and Sacco Attack}
\begin{itemize}
\item Message 3 is not protected by any freshness component;\pause
\item There is no way for B to know if the $K_{AB}$ it receives is current;\pause
\item Lack of freshness on message 3 means an intruder has unlimited time to crack an old session key and reuse it.
\end{itemize}
\end{frame}

\begin{frame}{Bauer, et al. Attack on NSSKP}
\begin{itemize}
\item Bauer, et al. pointed out that if key $K_{AS}$ were compromised, anyone could impersonate A and establish communication with any other party;\pause
\item Usually the lost of control on long term secrets affects deeply how a protocol operate;\pause
\item It is important to have mechanisms that could revoke keys or at least render them unusable after sometime.
\end{itemize}
\end{frame}

\begin{frame}{Questions!}
\begin{itemize}
\item These flaws persisted for almost 10 years before they were discovered. Why did it take that long to see that?\pause
\item Ask what happens if a key is broken is a fair question?\pause
\item How can you address these design faults pointed out by Denning and Sacco and Bauer et al.?
\end{itemize}
\end{frame}

\frame{
	\frametitle{Needham-Schroeder Public Key Protocol}
\begin{itemize}
\item Many existing protocols are derived from the work of Needham and Schroeder (1978);\pause 
\item The handshake done by many public key cryptographic protocols are derived from NSPKP;\pause
\item NSPKP is a public-key authentication protocol designed to generate and propagate a session key which is used for subsequent symmetrically encrypted communication;\pause
\item There is no public key infrastructure in place, but the identities related top public keys are an assumption.
\end{itemize}
}

\frame{\frametitle{Needham-Schroeder Public Key Protocol}

		\begin{table}[htdp]
\begin{center}
\begin{tabular}{ l l }
1. & A $\rightarrow$ B:  $\{| N_a , A |\}_{K_b}$ \\ \pause
2. & B $\rightarrow$ A:  $\{| N_a , N_b |\}_{K_a}$ \\ \pause
3. & A $\rightarrow$ B:  $\{| N_b |\}_{K_b}$ \\
\end{tabular}
\end{center}
\end{table}%
}

\begin{frame}{NSPKP Goals}
\begin{itemize}
\item The goal of the protocol is to establish mutual authentication between two parties A and B in the presence of adversary;\pause
\item A and B obtain a secret shared key though direct communication using public key cryptography;\pause
\item This adversary can intercept messages, delay messages, read and copy messages and generate messages;\pause
\item This adversary can not learn the privates keys of principals.
\end{itemize}
\end{frame}

\begin{frame}{Freshness Concepts of NSPKP}
\begin{itemize}
\item NSPKP uses nonces which are randomly generated values included in messages and used only once;\pause
\item If a nonce is generated and sent by one agent in one step and returned by another in a later step, the generator knows that the message is fresh and not a replay from an earlier exchange;\pause
\item Note that a nonce is not anchored in time. The only assumption is that it has not been used in any earlier interchange, with high probability because it is random and not used twice.
\end{itemize}
\end{frame}

\begin{frame}{Needham-Schroeder Public Key Protocol - Questions}
\begin{itemize}
\item What are the assumptions? \pause
\item What seems to be the goal?\pause 
\item What might the principals believe after each step?\pause
\item What is missing?\pause
\item Is it secure?
\end{itemize}
\end{frame}

\frame{\frametitle{Needham-Schroeder Public Key Protocol Description}

		\begin{table}[htdp]
\begin{center}
\begin{tabular}{ l l }
1. & A $\rightarrow$ B:  $\{| N_a , A |\}_{K_b}$ \\
& Alice send to Bob an encrypted nonce\\\pause
2. & B $\rightarrow$ A:  $\{| N_a , N_b |\}_{K_a}$ \\
& Bob returns to Alice her Nonce together with his own\\\pause
3. & A $\rightarrow$ B:  $\{| N_b |\}_{K_b}$ \\
& Alice returns to Bob his Nonce
\end{tabular}
\end{center}
\end{table}%
}




\frame{
	\frametitle{Needham-Schroeder Public Key Protocol Assumptions}
	\begin{table}[htdp]
\begin{center}
\begin{tabular}{ l l }
1. & A $\rightarrow$ B:  $\{| N_a , A |\}_{K_b}$ \\
& Only Alice knows Na before message 1\\
& Only Bob can decrypt message 1\\\pause
2. & B $\rightarrow$ A:  $\{| N_a , N_b |\}_{K_a}$ \\
& Only Bob knows Nb before message 2\\
& Bob knows Na because he can decrypt\\
& Ohly Alice can decrypt message 2\\\pause
3. & A $\rightarrow$ B:  $\{| N_b |\}_{K_b}$ \\
& Alice knows Nb because she can decrypt \\
& Only Bob can decrypt message 3\\\pause
&{\color{red}Why do we need message 3?}
\end{tabular}
\end{center}
\end{table}%
		
}

\frame{
	\frametitle{Needham-Schroeder Public Key Protocol Interpretation}
		\begin{table}[htdp]
\begin{center}
\begin{tabular}{ l l }
1. & A $\rightarrow$ B:  $\{| N_a , A |\}_{K_b}$ \\
& Alice starts a session Na is the session control\\
& Alice's identity (A) is for Bob to know whom \\
& to encrypt message 2\\\pause
2. & B $\rightarrow$ A:  $\{| N_a , N_b |\}_{K_a}$ \\
& Bom sens Na back to keep the session\\
&{\color{green} Bob uses Nb to authenticate Alice on message 3}\\
& When receiving message 2 Alice knows that only Bob could\\
&have created it because it contains Na\\\pause
3. & A $\rightarrow$ B:  $\{| N_b |\}_{K_b}$ \\
& Alice already authenticated Bob. Now she wants to authenticate to Bob\\
& When receiving message 3 Bob knows that only Alice could\\
& have created it because it contains Nb
\end{tabular}
\end{center}
\end{table}%
		
}

\frame{
	\frametitle{Needham-Schroeder Public Key Protocol Objectives}
\begin{table}[htdp]
\begin{center}
\begin{tabular}{ l l }
1. & A $\rightarrow$ B:  $\{| N_a , A |\}_{K_b}$ \\
2. & B $\rightarrow$ A:  $\{| N_a , N_b |\}_{K_a}$ \\
3. & A $\rightarrow$ B:  $\{| N_b |\}_{K_b}$ \\
\end{tabular}
\end{center}
\begin{itemize}
\item The protocols authenticates Alice to Bob;\pause
\item The protocols authenticate Bob to Alice;\pause
\item By the usage of fresh Nonces, we obtain the aliveness property in the protocols;\pause
\item {\color{red}But, Is it secure?}
\end{itemize}
\end{table}%
}

\frame{
	\frametitle{Almighty Attack - Rules of the Game}
\begin{itemize}
\item Charlie is a very powerful attacker. \pause
\item He can:\pause
\begin{itemize}
\item Intercept anything in the network;\pause
\item Block anything in the network;\pause
\item Replay Messages;\pause
\item Create new messages with what he learned with the traffic;\pause
\item Behave as a normal agent;\pause
\end{itemize}
\item He can not:\pause
\begin{itemize}
\item Break cryptography;\pause
\item Guess random numbers.
\end{itemize}
\end{itemize}

}


\frame{
	\frametitle{Almighty Attack - Lowe's Attack}
\begin{table}[htdp]
\begin{center}
\begin{tabular}{ l l }
1. & A $\rightarrow$ C:  $\{| N_a , A |\}_{K_c}$ \\\pause
{\color{red}1'. }& {\color{red}C(A) $\rightarrow$ B:  $\{| N_a , A |\}_{K_b}$ }\\
{\color{red}2'. }& {\color{red}B $\rightarrow$ C(A):  $\{| N_a , N_b |\}_{K_a}$ }\\\pause
2. & C $\rightarrow$ A:  $\{| N_a , N_b |\}_{K_a}$ \\\pause
3. & A $\rightarrow$ C:  $\{| N_b |\}_{K_c}$ \\\pause
{\color{red}3'. }& {\color{red}C $\rightarrow$ B:  $\{| N_b |\}_{K_b}$ }\\
\end{tabular}
\end{center}
\begin{itemize}
\item Bob believes to be talking to Alice , while he is talking to Charlie;\pause
\item Charlie uses Alice as an oracle to answers Bob's challenges;\pause
\item Charlie can use Nb to prove to Bob he is Alice.
\end{itemize}
\end{table}%
}


\frame{
	\frametitle{Lowe's Attack - Facts}
\begin{itemize}
\item Gavin Lowe was a random Computer Theoretician at Oxford;\pause
\item The attack looks easy, but it took 15 years to be found; \pause
\item The attack works because of a change on the threat model; \pause
\item But his attack is important because it was only discovered with the help of a formal verification tool.
\end{itemize}
}

\frame{
	\frametitle{How Lowe did it?}
\begin{itemize}
\item He used a tool called a model checker;\pause
\item This tool has able to saturate the protocol execution;\pause
\item The idea is to explore all the reachable states of the model;\pause
\item This verification is considered bound to the amount of peers and parallel runs tested.
\end{itemize}
}

\frame{
	\frametitle{What we learned from this example?}
\begin{itemize}
\item Some tests are too hard to be executed by hand;\pause
\item Theoretical tools are a good breakthrough for any area;\pause
\item Even very simple and well studied protocols may contain hidden failures;\pause
\item We learned to be diligent and somewhat paranoid on protocols and how they achieve their goals.
\end{itemize}
}

\frame{
	\frametitle{Discussion}
\begin{itemize}
\item Which other protocols do you believe there are failures that were not detected yet?\pause
\item How do we correct Lowe's attack on NSPKP?\pause
\item Is formally proving a protocol enough to claim it is secure?\pause
\item What if a user drop an assumption of the protocol? Is it still secure?\pause
\item How secure is formally secure?
\end{itemize}
}

{
\usebackgroundtemplate{\includegraphics[width=\paperwidth,
height=\paperheight]{../reusable_images/fundo_capa.png}}
\begin{frame}

{\LARGE Questions????}

\end{frame}
}

{
\usebackgroundtemplate{\includegraphics[width=\paperwidth,
height=\paperheight]{../reusable_images/fundo_capa.png}}
\begin{frame}
\includegraphics[scale=0.8]{../reusable_images/cc_logo_arge.png}\hspace{0.5cm} 
\includegraphics[scale=0.95]{../reusable_images/by.png}

\vspace{1cm}
This work is licensed under the Creative Commons Attribution 4.0 International License. To view a copy of this license, visit http://creativecommons.org/licenses/by/4.0/.
\end{frame}
}

\end{document}