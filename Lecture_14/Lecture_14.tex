\documentclass[12pt,table,xcolor={dvipsnames}]{beamer}
\usetheme{Pittsburgh}
\usecolortheme{seagull}
%\usepackage[utf8]{inputenc}
\usepackage{fontspec}
\usepackage{amsmath}
\usepackage{listings}
\usepackage{multirow}
\usepackage{amsfonts}
\usepackage{amssymb}
\usepackage{graphicx}
\author{Design and Verification of Security Protocols and Security Ceremonies}
\title{Protocol Verification Techniques - State Enumeration - Continuation}
%\setbeamercovered{transparent} 
\setbeamertemplate{navigation symbols}{} 
%\logo{\includegraphics[scale=0.015]{Brasao_UFSC.png}\includegraphics[scale=0.2]{brasao_PPGCC.jpg}} 
\institute{Programa de Pós-Graduacão em Ciências da Computacão \\ Dr. Jean Everson Martina} 
\date{\vspace{March-June 2019} 
\subject{} 
\usebackgroundtemplate{\includegraphics[width=\paperwidth,
height=\paperheight]{../reusable_images/fundo_UFSC.png}}
\begin{document}

{
\usebackgroundtemplate{\includegraphics[width=\paperwidth,
height=\paperheight]{../reusable_images/fundo_capa.png}}
\begin{frame}
\titlepage
\includegraphics[scale=0.3]{../reusable_images/brasao_PPGCC.jpg}
\end{frame}
}

\frame{
	\frametitle{Before we start!}
\begin{block}{Attention!}
This lecture is a continuation of the previous lecture. We will start with a quick glimpse of what need to be remembered to continue with the contents.
\end{block}
}


\frame{
	\frametitle{CSP Primitives}
\begin{itemize}
\item CSP provides two classes of primitives in its process algebra:
\item Events:
\begin{itemize}
\item Events represent communications or interactions;
\item Events are assumed to be indivisible and instantaneous;
\item They may be atomic names, compound names, or input/output events;
\end{itemize}
\item Primitive processes:
\begin{itemize}
\item Primitive processes represent fundamental behaviours;
\item STOP (the process that communicates nothing, also called deadlock);
\item SKIP (which represents successful termination).
\end{itemize}
\end{itemize}
} 

\frame{
	\frametitle{CSP Algebraic Operators}
${\displaystyle {\begin{array}{lcll}
Proc&::=&{\textit {STOP}}&\;\\
&|&{\textit {SKIP}}&\;\\
&|&e\rightarrow {\textit {Proc}}&({\text{prefixing}})\\
&|&{\textit {Proc}}\;\Box \;{\textit {Proc}}&({\text{external}}\;{\text{choice}})\\
&|&{\textit {Proc}}\;\sqcap \;{\textit {Proc}}&({\text{nondeterministic}}\;{\text{choice}})\\
&|&{\textit {Proc}}\;\vert \vert \vert \;{\textit {Proc}}&({\text{interleaving}})\\
&|&{\textit {Proc}}\;|[\{X\}]|\;{\textit {Proc}}&({\text{interface}}\;{\text{parallel}})\\
&|&{\textit {Proc}}\setminus X&({\text{hiding}})\\
&|&{\textit {Proc}};{\textit {Proc}}&({\text{sequential}}\;{\text{composition}})\\
&|&\mathrm {if} \;b\;\mathrm {then} \;{\textit {Proc}}\;\mathrm {else} \;Proc&({\text{boolean}}\;{\text{conditional}})\\
&|&{\textit {Proc}}\;\triangleright \;{\textit {Proc}}&({\text{timeout}})\\
&|&{\textit {Proc}}\;\triangle \;{\textit {Proc}}&({\text{interrupt}})\end{array}}}$
} 

\frame{\frametitle{Needham-Schroeder Public Key Protocol}

		\begin{table}[htdp]
\begin{center}
\begin{tabular}{ l l }
1. & A $\rightarrow$ B:  $\{| N_a , A |\}_{K_b}$ \\ 
2. & B $\rightarrow$ A:  $\{| N_a , N_b |\}_{K_a}$ \\ 
3. & A $\rightarrow$ B:  $\{| N_b |\}_{K_b}$ \\
\end{tabular}
\end{center}
\end{table}%
}

\begin{frame}{NSPKP Goals}
\begin{itemize}
\item The goal of the protocol is to establish mutual authentication between two parties A and B in the presence of adversary;
\item A and B obtain a secret shared key though direct communication using public key cryptography;
\item This adversary can intercept messages, delay messages, read and copy messages and generate messages;
\item This adversary can not learn the privates keys of principals.
\end{itemize}
\end{frame}

\frame{
	\frametitle{Lowe's Specification using CSP}
\begin{itemize}
\item $MSG1 \equiv \{ Msg1.a.b.Encrypt.k.n_a.a' |$\\$ a \in Initiator, a' \in Initiator, b \in Responder,$\\$k \in Key, n_a \in Nonce\},$
\item $MSG2 \equiv \{Msg2.b.a.Encrypt.k.n_a.n_b |$\\$
a \in Initiator, b \in Responder,$\\$
k \in Key, n_a \in Nonce, n_b \in Nonce\},$
\item $MSG3 \equiv \{Msg3.a.b.Encrypt.k.n_b |$\\$
a \in Initiator, b \in Responder, k \in Key, n_b \in Nonce\},$
\item $MSG \equiv MSG1 \cup MSG2 \cup MSG3. $
\end{itemize}
} 

\frame{
	\frametitle{Lowe's Specification using CSP}
\begin{itemize}
\item Standard communications in the system will be modelled by the channel comm;
\item We also want to model the fact that the intruder can fake or intercept messages, and so we introduce extra channels fake and intercept;
\item ${\displaystyle channel \quad comm, fake, intercept : MSG.} $
\item This will ensure that the receiver of a faked message is not aware that it is a fake, and that the sender of an intercepted message is not aware that it is intercepted. 
\end{itemize}
} 

\frame{
	\frametitle{Lowe's Specification using CSP}
\begin{itemize}
\item We introduce two extra channels, defining the external interface of the protocol; 
\item We represent a request from a user for initiator ''a" to connect with responder ''b" by the event $user.a.b$; 
\item We represent the resulting session by the event $session.a.b$;
\end{itemize}
} 

\frame{
	\frametitle{Lowe's Specification using CSP}
\begin{itemize}
\item We also add channels to represent the state of the agents:
\begin{itemize}
\item We represent the initiator ''a" thinking it is taking part in a run of the protocol with ''b"
by the event $I\_running.a.b$;
\item We represent the responder ''b" thinking it is taking part in a run of the protocol with ''a" by the event $R\_running.a.b$; 
\item We represent the initiator committing to the session by the $I\_commit.a.b$;
\item We represent the responder committing to the session by $R\_commit.a.b$;
\end{itemize}
\item We declare these channels by: 
\item  ${\displaystyle channel \quad user, session, I\_running, R\_running,}$\\${\displaystyle  I\_commit, R\_.commit : Initiator.Responder.} $
\end{itemize}
} 

\frame{
	\frametitle{Lowe's Specification using CSP}
\begin{itemize}
\item We will represent a responder with identity ''a", who has a single nonce $n_a$, by the CSP process $INITIATOR(a, na)$.;\pause
\item If we want to consider a responder with more than one nonce, then we can compose several such processes, either sequentially or interleaved;\pause
\item Ignoring, for the moment, the possibility of intruder action, the process can be defined by: 
\begin{itemize}
\item  ${\displaystyle INITIATOR(a, n_a) \equiv  user.a?b \rightarrow I\_.running.a.b \rightarrow}$\\
 ${\displaystyle comm!Msg1.a.b.Encrypt.key( b ) .n_a,a \rightarrow}$\\
 ${\displaystyle comm.Msg2.b.a.Encrypt.key( a) ?n_a'.n_b \rightarrow}$\\
 ${\displaystyle if \quad n_a = n_a'}$\\
 ${\displaystyle \quad then \quad comm!Msg3.a.b.Encrypt.key(b).n_b \rightarrow}$\\
 ${\displaystyle \quad I\_commit.a.b \rightarrow session.a.b \rightarrow SKIP}$\\
 ${\displaystyle else \quad STOP. } $
\end{itemize}
\end{itemize}
} 

\frame{
	\frametitle{Lowe's Specification using CSP}
\begin{itemize}
\item We now introduce the possibility of enemy action by applying a renaming to the above process;\pause 
\item Our renaming should ensure that message ls and message 3s sent by the initiator can be intercepted;\pause
\item And message 2s can be faked;\pause
\item We define an initiator with identity A and nonce Na by:
${\displaystyle INITIATOR1 \equiv INITIATOR(A, N_a)}$\\
${\displaystyle [[comm.Msg1 \leftarrow comm.Msg1 , comm.Msg1 \leftarrow intercept.Msg1,}$\\
${\displaystyle comm.Msg2 \leftarrow comm.Msg2, comm.Msg2 \leftarrow fake.Msg2,}$\\
${\displaystyle comm.Msg3 \leftarrow comm.Msg3, comm.Msg3 \leftarrow intercept.Msg3]].}$\\
\end{itemize}
} 

\frame{
	\frametitle{Slowing Down!}
\begin{block}{Slowing Down!}
From this point on we will start to slow down because the content thickens...
\end{block}
}

\frame{
	\frametitle{Lowe's Intruder}
\begin{itemize}
\item The intruder should be able to:\pause
\begin{itemize}
\item Overhear and/or intercept any messages being passed in the system;\pause
\item Decrypt messages that are encrypted with his own public key, so as to learn new nonces;\pause
\item Introduce new messages into the system, using nonces he knows;\pause
\item Replay any message he has seen (possibly changing plain-text parts), even if he does not understand the contents of the encrypted part;
\end{itemize}
\end{itemize}
} 

\frame{
	\frametitle{Lowe's Intruder}
\begin{itemize}
\item Replay any message he has seen (possibly changing plain-text parts), even if he does not understand the contents of the encrypted part;\pause
\item We assume that the intruder is a user of the computer network;\pause
\item  We will define the most general (i.e. the most non-deterministic) intruder who can act as above;\pause
\item We consider an intruder with identity I, with public key $K_i$, who initially knows a nonce $N_i$. 
\end{itemize}
} 

\frame{
	\frametitle{Lowe's Intruder Definition in CSP}
${\displaystyle I(m1s, m2s, m3s, ns)}$\\\pause
${\displaystyle \quad comm.Msg1 ? a.b.Encrypt.k.n.a' \rightarrow}$\\
${\displaystyle \quad if \quad k = K_i \quad then \quad I(m1s, m2s, m3s,ns \cup \{n\})}$\\
${\displaystyle \quad else \quad I(m1s \cup \{ Encrypt.k.n.a'\}, m2s, m3s, ns)}$\\\pause
${\displaystyle \quad \Box  intercept.Msg1 ? a.b.Encrypt.k.n.a' \rightarrow}$\\
${\displaystyle \quad if \quad k = K_i \quad  then \quad I(m1s,m2s, m3s,ns \cup \{n\})}$\\
${\displaystyle \quad else \quad I(m1s \cup \{ Encrypt.k.n.a'\}, m2s, m3s, ns)}$\\\pause
${\displaystyle \quad \Box  comm.Msg2?b.a.Encrypt.k.n.n' \rightarrow}$\\
${\displaystyle \quad if \quad k = K_i \quad then \quad I(m1s, m2s, m3s, ns \cup \{n, n'\})}$\\
${\displaystyle \quad else \quad I(m1s, m2s \cup { Encrypt.k,n.n'}, m3s, ns)}$\\\pause
${\displaystyle \quad \Box  intercept.Msg2?b.a.Encrypt.k.n.n' \rightarrow}$\\
${\displaystyle \quad if \quad k = K_i \quad then \quad I(m1s,m2s, m3s,ns \cup \{n,n'\})}$\\
${\displaystyle \quad else I(m1s, m2s \cup { Encrypt.k.n.n'}, m3s, ns)}$\\
} 

\frame{
	\frametitle{Lowe's Intruder Definition in CSP}
${\displaystyle \quad \Box   comm.Msg3? a.b.Encrypt.k.n \rightarrow}$\\
${\displaystyle \quad if \quad k = K_i \quad then \quad I(m1s,m2s, m3s, ns \cup \{n\})}$\\
${\displaystyle \quad else I(m ls, m2s, rn3s \cup { Encrypt.k.n} , ns)}$\\\pause
${\displaystyle\quad \Box   intercept.Msg3? a.b.Encrypt.k.n \rightarrow}$\\
${\displaystyle \quad if \quad k = K_i \quad then \quad I(m1s, m2s, m3s, ns \cup \{n\})}$\\
${\displaystyle \quad else I(m1s, m2s, m3s \cup { Encrypt.k.n}, ns)}$\\\pause
${\displaystyle \quad \Box  fake.Msg1?a.b?m :m1s \rightarrow I(m1s, m2s, m3s, ns)}$\\
${\displaystyle \quad \Box  fake.Msg2?a.b?m :m2s \rightarrow I(m1s, m2s, m3s, ns)}$\\
${\displaystyle \quad \Box  fake.Msg3?a.b?m :m3s \rightarrow I(m1s, m2s, m3s, ns)}$\\\pause
${\displaystyle \quad \Box  fake.Msg1?a.blEncrypt?k?n:ns?a' \rightarrow I(m1s, m2s, m3s, ns)}$\\
${\displaystyle \quad \Box  fake.Msg2?b.a!Encrypt?k?n:ns?n':ns \rightarrow I(m1s, m2s, m3s, ns)}$\\
${\displaystyle \quad \Box  fake.Msg3?a.b!Encrypt?k?n:ns \rightarrow I(m1s, m2s, m3s, ns). }$\\
} 

\frame{
	\frametitle{Lowe's Intruder Definition in CSP}
\begin{itemize}
\item We consider an intruder who initially knows the nonce $N_i$:
\begin{itemize}
\item ${\displaystyle INTRUDER \equiv I(\{\}, \{\}, \{\}, \{N_i\}).}$\pause
\end{itemize}
\item We may now define a system with an intruder:
\begin{itemize}
\item ${\displaystyle AGENTS \equiv}$\\
${\displaystyle INITIATOR1 |[ \{|comm, session.A.B|\}] |RESPONDER1,}$\\
${\displaystyle SYSTEM \equiv AGENTS |[ \{|fake, comm, intercept |\} ]| INTRUDER. }$\\
\end{itemize}
\end{itemize}
}

\frame{
	\frametitle{Refining CSP with FDR}
\begin{itemize}
\item FDR takes as two inputs, a specification and an implementation, and test whether the implementation refines the specification;\pause
\item  We are working in the traces model of CSP, so checking for refinement amounts to testing whether each trace of the implementation is also a trace of the specification;
\end{itemize}
} 

\frame{
	\frametitle{Testing NSPKP with FDR}
\begin{itemize}
\item To test whether the protocol correctly authenticates the responder, we need to find a specification that allows only those traces where the initiator A commits to a session with B only if B has indeed taken part in the protocol run;\pause
\item The initiator committing to a session is represented by an I\_commit.A.B event; the responder taking part in a run of the protocol with A is represented by R\_running.A.B;\pause
\item Thus the authenticity of the responder can be tested using the specification AR:
\begin{itemize}
\item ${\displaystyle AR_0 \equiv R\_running.A.B \rightarrow I\_commit.A.B \rightarrow AR_0,}$\\
${\displaystyle A1 \equiv \{|R\_running.A.B , I\_commit.A.B|\} ,}$\\
${\displaystyle AR \equiv AR_0 ||| RUN(\sum }$ \ ${\displaystyle A1).}$\\
\end{itemize} 
\end{itemize}
} 

\frame{
	\frametitle{Testing NSPKP with FDR}
\begin{itemize}
\item We now consider authentication of the initiator. The protocol should ensure that the responder B commits to a session with initiator A only if A took part in the protocol run; \pause
\item Formally, an R\_commit.A.B event should occur only if there has been a corresponding I\_running.A.B event; \pause
\item This requirement is captured by the specification AI:
\begin{itemize}
\item ${\displaystyle AI_0 \equiv I\_running.A.B \rightarrow R_commit.A.B \rightarrow AI_0,}$\\
${\displaystyle A_2 \equiv \{|I\_running.A.B,R\_commit.A.B|\},}$\\
${\displaystyle AI \equiv AI_0 ||| RUN(\sum }$ \ ${\displaystyle A2).}$\\
\end{itemize} 
\end{itemize}
} 

\frame{
	\frametitle{Testing NSPKP with FDR - the Flaw}
\begin{itemize}
\item FDR can be used to discover that SYSTEM does not refine AI;\pause
\item It finds that after the trace:
\begin{itemize}
\item ${\displaystyle (user.A.I , I\_running.A.I ,}$\\
${\displaystyle intercept.Msg1.A.I.Encrypt.K_i.N_a.A ,}$\\
${\displaystyle fake.Msgl.A.B.Encrypt.K_b.N_a.A ,}$\\
${\displaystyle intercept.Msg2.B.A.Encrypt.K_a.N_a.N_b ,}$\\
${\displaystyle fake.Msg2.I.A.Encrypt.K_a.N_a.N_b ,}$\\
${\displaystyle intercept.Msg3.A.I.Encrypt.K_i.N_b ,}$\\
${\displaystyle fake.Msg3.A.B.Encrypt.K_b.N_b) }$\\
\end{itemize} 
\end{itemize}
} 

\frame{
	\frametitle{Lowe's Attack - Translated}
\begin{table}[htdp]
\begin{center}
\begin{tabular}{ l l }
1. & A $\rightarrow$ C:  $\{| N_a , A |\}_{K_c}$ \\
{\color{red}1'. }& {\color{red}C(A) $\rightarrow$ B:  $\{| N_a , A |\}_{K_b}$ }\\
{\color{red}2'. }& {\color{red}B $\rightarrow$ C(A):  $\{| N_a , N_b |\}_{K_a}$ }\\
2. & C $\rightarrow$ A:  $\{| N_a , N_b |\}_{K_a}$ \\
3. & A $\rightarrow$ C:  $\{| N_b |\}_{K_c}$ \\
{\color{red}3'. }& {\color{red}C $\rightarrow$ B:  $\{| N_b |\}_{K_b}$ }\\
\end{tabular}
\end{center}
\begin{itemize}
\item Bob believes to be talking to Alice , while he is talking to Charlie;\pause
\item Charlie uses Alice as an oracle to answers Bob's challenges;\pause
\item Charlie can use Nb to prove to Bob he is Alice.
\end{itemize}
\end{table}%
}
 
\frame{
	\frametitle{Discussion}
\begin{itemize}
\item Would Lowe be able to find this attack by hand?\pause
\item What else could be testes using this strategy?\pause
\item Was it the methodology or the Threat Model?
\end{itemize}
}

{
\usebackgroundtemplate{\includegraphics[width=\paperwidth,
height=\paperheight]{../reusable_images/fundo_capa.png}}
\begin{frame}

{\LARGE Questions????}

\end{frame}
}

{
\usebackgroundtemplate{\includegraphics[width=\paperwidth,
height=\paperheight]{../reusable_images/fundo_capa.png}}
\begin{frame}
\includegraphics[scale=0.8]{../reusable_images/cc_logo_arge.png}\hspace{0.5cm} 
\includegraphics[scale=0.95]{../reusable_images/by.png}

\vspace{1cm}
This work is licensed under the Creative Commons Attribution 4.0 International License. To view a copy of this license, visit http://creativecommons.org/licenses/by/4.0/.
\end{frame}
}

\end{document}