\documentclass[12pt,table,xcolor={dvipsnames}]{beamer}
\usetheme{Pittsburgh}
\usecolortheme{seagull}
%\usepackage[utf8]{inputenc}
\usepackage{fontspec}
\usepackage{amsmath}
\usepackage{listings}
\usepackage{multirow}
\usepackage{amsfonts}
\usepackage{amssymb}
\usepackage{graphicx}
\author{Design and Verification of Security Protocols and Security Ceremonies}
\title{\vspace{-.2cm}Protocol Verification Techniques - State Enumeration}
%\setbeamercovered{transparent} 
\setbeamertemplate{navigation symbols}{} 
%\logo{\includegraphics[scale=0.015]{Brasao_UFSC.png}\includegraphics[scale=0.2]{brasao_PPGCC.jpg}} 
\institute{Programa de Pós-Graduacão em Ciências da Computacão \\ Dr. Jean Everson Martina} 
\date{\vspace{-1cm}August-November 2016} 
\subject{} 
\usebackgroundtemplate{\includegraphics[width=\paperwidth,
height=\paperheight]{../reusable_images/fundo_UFSC.png}}
\begin{document}

{
\usebackgroundtemplate{\includegraphics[width=\paperwidth,
height=\paperheight]{../reusable_images/fundo_capa.png}}
\begin{frame}
\titlepage
\includegraphics[scale=0.3]{../reusable_images/brasao_PPGCC.jpg}
\end{frame}
}

\frame{
	\frametitle{Before we start!}
\begin{block}{Before we start!}
To be able to talk about Lowe's attack using FDR and CSP, we need first to get the grips with the theory behind communicating sequential processes (CSP).
\end{block}
}

\frame{
	\frametitle{Communicating Sequential Processes}
\begin{itemize}
\item Communicating Sequential Processes (CSP) is a formal language for describing patterns of interaction in concurrent systems;\pause
\item It is a member of the family of mathematical theories of concurrency known as process algebras;\pause
\item CSP was first described in a 1978 paper by Tony Hoare;\pause
\item CSP has been practically applied in industry as a tool for specifying and verifying the concurrent aspects of a variety of different systems.
\end{itemize}
}


\frame{
	\frametitle{Communicating Sequential Processes}
\begin{itemize}
\item The theory of CSP itself is also still the subject of active research, including work to increase its range of practical applicability;\pause
\item CSP allows the description of systems in terms of component processes that operate independently, and interact with each other solely through message-passing communication;\pause
\item  The relationships between different processes, and the way each process communicates with its environment, are described using various process algebraic operators;\pause
\item Using this algebraic approach, quite complex process descriptions can be easily constructed from a few primitive elements.
\end{itemize}
}
 
\frame{
	\frametitle{CSP Primitives}
\begin{itemize}
\item CSP provides two classes of primitives in its process algebra:\pause
\item Events:\pause
\begin{itemize}
\item Events represent communications or interactions;\pause
\item Events are assumed to be indivisible and instantaneous;\pause
\item They may be atomic names, compound names, or input/output events;\pause
\end{itemize}
\item Primitive processes:\pause
\begin{itemize}
\item Primitive processes represent fundamental behaviours;\pause
\item STOP (the process that communicates nothing, also called deadlock);\pause
\item SKIP (which represents successful termination).
\end{itemize}
\end{itemize}
} 

\frame{
	\frametitle{CSP Algebraic Operators}
${\displaystyle {\begin{array}{lcll}
Proc&::=&{\textit {STOP}}&\;\\
&|&{\textit {SKIP}}&\;\\\pause
&|&e\rightarrow {\textit {Proc}}&({\text{prefixing}})\\\pause
&|&{\textit {Proc}}\;\Box \;{\textit {Proc}}&({\text{external}}\;{\text{choice}})\\\pause
&|&{\textit {Proc}}\;\sqcap \;{\textit {Proc}}&({\text{nondeterministic}}\;{\text{choice}})\\\pause
&|&{\textit {Proc}}\;\vert \vert \vert \;{\textit {Proc}}&({\text{interleaving}})\\\pause
&|&{\textit {Proc}}\;|[\{X\}]|\;{\textit {Proc}}&({\text{interface}}\;{\text{parallel}})\\\pause
&|&{\textit {Proc}}\setminus X&({\text{hiding}})\\\pause
&|&{\textit {Proc}};{\textit {Proc}}&({\text{sequential}}\;{\text{composition}})\\\pause
&|&\mathrm {if} \;b\;\mathrm {then} \;{\textit {Proc}}\;\mathrm {else} \;Proc&({\text{boolean}}\;{\text{conditional}})\\\pause
&|&{\textit {Proc}}\;\triangleright \;{\textit {Proc}}&({\text{timeout}})\\\pause
&|&{\textit {Proc}}\;\triangle \;{\textit {Proc}}&({\text{interrupt}})\end{array}}}$
} 

\frame{
	\frametitle{CSP Example - Vending Machine}
\begin{itemize}
\item One of the archetypal CSP examples is an abstract representation of a chocolate vending machine and its interactions with a person wishing to buy some chocolate;\pause
\item This vending machine might be able to carry out two different events, “coin” and “choc”;\pause
\item They which represent the insertion of payment and the delivery of a chocolate respectively. 
\end{itemize}
} 

\frame{
	\frametitle{CSP Example - Vending Machine}
\begin{itemize}
\item A machine which demands payment before offering a chocolate can be written as:
\begin{itemize}\pause
\item ${\displaystyle {\textit {VendingMachine}}={\textit {coin}}\rightarrow {\textit {choc}}\rightarrow {\textit {STOP}}}$\pause
\end{itemize}
\item A person who might choose to use a coin or card to make payments could be modelled as:\pause
\begin{itemize}
\item ${\displaystyle {\textit {Person}}=({\textit {coin}}\rightarrow {\textit {STOP}})\Box ({\textit {card}}\rightarrow {\textit {STOP}})})$\pause
\end{itemize}
\item These two processes can be put in parallel, so that they can interact with each other;\pause
\item The behaviour of the composite process depends on the events that the two processes must synchronise on:\pause
\begin{itemize}
\item ${\displaystyle {\textit {VendingMachine}}\left\vert \left[\left\{{\textit {coin}},{\textit {card}}\right\}\right]\right\vert {\textit {Person}}\equiv} $\\${\displaystyle {\textit {coin}}\rightarrow {\textit {choc}}\rightarrow {\textit {STOP}}}$
\end{itemize}
\end{itemize}
} 

\frame{
	\frametitle{CSP Example - Vending Machine}
\begin{itemize}

\item If synchronization was only required on “coin”, we would obtain:\pause
\begin{itemize}
\item ${\displaystyle {\textit {VendingMachine}}\left\vert \left[\left\{{\textit {coin}}\right\}\right]\right\vert {\textit {Person}}\equiv}$ \\ ${\displaystyle \left({\textit {coin}}\rightarrow {\textit {choc}}\rightarrow {\textit {STOP}}\right)\Box \left({\textit {card}}\rightarrow {\textit {STOP}}\right)}$\pause
\end{itemize}
\item If we abstract this latter composite process by hiding the “coin” and “card” events:\pause
\begin{itemize}
\item ${\displaystyle \left(\left({\textit {coin}}\rightarrow {\textit {choc}}\rightarrow {\textit {STO}}P\right)\Box \left({\textit {card}}\rightarrow {\textit {STOP}}\right)\right)}$ \\${\displaystyle  \setminus \left\{{\textit {coin}},card\right\}}$\pause
\end{itemize}
\item We get the non-deterministic process:\pause
\begin{itemize}
\item ${\displaystyle \left({\textit {choc}}\rightarrow {\textit {STOP}}\right)\sqcap {\textit {STOP}}}$\pause
\end{itemize}
\item This process which either offers a “choc” event and then stops, or just stops;\pause
\item Externally non-determinism has been introduced.
\end{itemize}
} 

\frame{\frametitle{Needham-Schroeder Public Key Protocol}

		\begin{table}[htdp]
\begin{center}
\begin{tabular}{ l l }
1. & A $\rightarrow$ B:  $\{| N_a , A |\}_{K_b}$ \\ \pause
2. & B $\rightarrow$ A:  $\{| N_a , N_b |\}_{K_a}$ \\ \pause
3. & A $\rightarrow$ B:  $\{| N_b |\}_{K_b}$ \\
\end{tabular}
\end{center}
\end{table}%
}

\begin{frame}{NSPKP Goals}
\begin{itemize}
\item The goal of the protocol is to establish mutual authentication between two parties A and B in the presence of adversary;\pause
\item A and B obtain a secret shared key though direct communication using public key cryptography;\pause
\item This adversary can intercept messages, delay messages, read and copy messages and generate messages;\pause
\item This adversary can not learn the privates keys of principals.
\end{itemize}
\end{frame}


\frame{
	\frametitle{Lowe's Specification using CSP}
\begin{itemize}
\item $MSG1 \equiv \{ Msg1.a.b.Encrypt.k.n_a.a' |$\\$ a \in Initiator, a' \in Initiator, b \in Responder,$\\$k \in Key, n_a \in Nonce\},$\pause
\item $MSG2 \equiv \{Msg2.b.a.Encrypt.k.n_a.n_b |$\\$
a \in Initiator, b \in Responder,$\\$
k \in Key, n_a \in Nonce, n_b \in Nonce\},$\pause
\item $MSG3 \equiv \{Msg3.a.b.Encrypt.k.n_b |$\\$
a \in Initiator, b \in Responder, k \in Key, n_b \in Nonce\},$\pause
\item $MSG \equiv MSG1 \cup MSG2 \cup MSG3. $
\end{itemize}
} 

\frame{
	\frametitle{Lowe's Specification using CSP}
\begin{itemize}
\item Standard communications in the system will be modelled by the channel comm;\pause
\item We also want to model the fact that the intruder can fake or intercept messages, and so we introduce extra channels fake and intercept;\pause
\item ${\displaystyle channel \quad comm, fake, intercept : MSG.} $\pause
\item This will ensure that the receiver of a faked message is not aware that it is a fake, and that the sender of an intercepted message is not aware that it is intercepted. 
\end{itemize}
} 

\frame{
	\frametitle{Lowe's Specification using CSP}
\begin{itemize}
\item We introduce two extra channels, defining the external interface of the protocol;\pause 
\item We represent a request from a user for initiator ''a" to connect with responder ''b" by the event $user.a.b$; \pause
\item We represent the resulting session by the event $session.a.b$;\pause
\end{itemize}
} 

\frame{
	\frametitle{Lowe's Specification using CSP}
\begin{itemize}
\item We also add channels to represent the state of the agents:\pause
\begin{itemize}
\item We represent the initiator ''a" thinking it is taking part in a run of the protocol with ''b"
by the event $I\_running.a.b$;\pause
\item We represent the responder ''b" thinking it is taking part in a run of the protocol with ''a" by the event $R\_running.a.b$; \pause
\item We represent the initiator committing to the session by the $I\_commit.a.b$;\pause
\item We represent the responder committing to the session by $R\_commit.a.b$;\pause
\end{itemize}
\item We declare these channels by: 
\item  ${\displaystyle channel \quad user, session, I\_running, B\_running,}$\\${\displaystyle  I\_commit, B\_.commit : Initiator.Responder.} $
\end{itemize}
} 

\frame{
	\frametitle{Lowe's Specification using CSP}
\begin{itemize}
\item We will represent a responder with identity ''a", who has a single nonce $n_a$, by the CSP process $INITIATOR(a, na)$.;\pause
\item If we want to consider a responder with more than one nonce, then we can compose several such processes, either sequentially or interleaved;\pause
\item Ignoring, for the moment, the possibility of intruder action, the process can be defined by: 
\begin{itemize}
\item  ${\displaystyle INITIATOR(a, n_a) \equiv  user.a?b \rightarrow I\_.running.a.b \rightarrow}$\\
 ${\displaystyle comm!Msg1.a.b.Encrypt.key( b ) .n_a,a \rightarrow}$\\
 ${\displaystyle comm.Msg2.b.a.Encrypt.key( a) ?n_a'.n_b \rightarrow}$\\
 ${\displaystyle if \quad n_a = n_a'}$\\
 ${\displaystyle \quad then \quad comm!Msg3.a.b.Encrypt.key(b).n_b \rightarrow}$\\
 ${\displaystyle \quad I\_commit.a.b \rightarrow session.a.b \rightarrow SKIP}$\\
 ${\displaystyle else \quad STOP. } $
\end{itemize}
\end{itemize}
} 

\frame{
	\frametitle{Lowe's Specification using CSP}
\begin{itemize}
\item We now introduce the possibility of enemy action by applying a renaming to the above process;\pause 
\item Our renaming should ensure that message ls and message 3s sent by the initiator can be intercepted;\pause
\item And message 2s can be faked;\pause
\item We define an initiator with identity A and nonce Na by:
${\displaystyle INITIATOR1 \equiv INITIATOR(A, N_a)}$\\
${\displaystyle [[comm.Msg1 \leftarrow comm.Msg1 , comm.Msg1 \leftarrow intercept.Msg1,}$\\
${\displaystyle comm.Msg2 \leftarrow comm.Msg2, comm.Msg2 \leftarrow fake.Msg2,}$\\
${\displaystyle comm.Msg3 \leftarrow comm.Msg3, comm.Msg3 \leftarrow intercept.Msg3]].}$\\
\end{itemize}
} 

\frame{
	\frametitle{Lowe's Intruder}
\begin{itemize}
\item The intruder should be able to:\pause
\begin{itemize}
\item Overhear and/or intercept any messages being passed in the system;\pause
\item Decrypt messages that are encrypted with his own public key, so as to learn new nonces;\pause
\item Introduce new messages into the system, using nonces he knows;\pause
\item Replay any message he has seen (possibly changing plain-text parts), even if he does not understand the contents of the encrypted part;\pause
\end{itemize}
\end{itemize}
} 

\frame{
	\frametitle{Lowe's Intruder}
\begin{itemize}
\item Replay any message he has seen (possibly changing plain-text parts), even if he does not understand the contents of the encrypted part;\pause
\item We assume that the intruder is a user of the computer network;\pause
\item  We will define the most general (i.e. the most non-deterministic) intruder who can act as above;\pause
\item We consider an intruder with identity I, with public key $K_i$, who initially knows a nonce $N_i$. 
\end{itemize}
} 

\frame{
	\frametitle{Lowe's Intruder}
\begin{block}{To be continued....}
On next lecture we will follow from here...
\end{block}
} 


{
\usebackgroundtemplate{\includegraphics[width=\paperwidth,
height=\paperheight]{../reusable_images/fundo_capa.png}}
\begin{frame}

{\LARGE Questions????}

\end{frame}
}

{
\usebackgroundtemplate{\includegraphics[width=\paperwidth,
height=\paperheight]{../reusable_images/fundo_capa.png}}
\begin{frame}
\includegraphics[scale=0.8]{../reusable_images/cc_logo_arge.png}\hspace{0.5cm} 
\includegraphics[scale=0.95]{../reusable_images/by.png}

\vspace{1cm}
This work is licensed under the Creative Commons Attribution 4.0 International License. To view a copy of this license, visit http://creativecommons.org/licenses/by/4.0/.
\end{frame}
}

\end{document}