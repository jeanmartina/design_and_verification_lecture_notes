\documentclass[12pt,table,xcolor={dvipsnames}]{beamer}
\usetheme{Pittsburgh}
\usecolortheme{seagull}
%\usepackage[utf8]{inputenc}
\usepackage{fontspec}
\usepackage{amsmath}
\usepackage{listings}
\usepackage{multirow}
\usepackage{amsfonts}
\usepackage{amssymb}
\usepackage{graphicx}
\author{Design and Verification of Security Protocols and Security Ceremonies}
\title{\vspace{-.2cm}Advanced Security Properties and Properties Composition}
%\setbeamercovered{transparent} 
\setbeamertemplate{navigation symbols}{} 
%\logo{\includegraphics[scale=0.015]{Brasao_UFSC.png}\includegraphics[scale=0.2]{brasao_PPGCC.jpg}} 
\institute{Programa de Pós-Graduacão em Ciências da Computacão \\ Dr. Jean Everson Martina} 
\date{\vspace{-1cm}August-November 2016} 
\subject{} 
\usebackgroundtemplate{\includegraphics[width=\paperwidth,
height=\paperheight]{../reusable_images/fundo_UFSC.png}}
\begin{document}

{
\usebackgroundtemplate{\includegraphics[width=\paperwidth,
height=\paperheight]{../reusable_images/fundo_capa.png}}
\begin{frame}
\titlepage
\includegraphics[scale=0.3]{../reusable_images/brasao_PPGCC.jpg}
\end{frame}
}

\begin{frame}{Disclaimer}
\begin{block}{Disclaimer!}
What we will see in the next slides is not a precise scientific description of the properties, but a basis for exemplification of what is out there just to foster discussion.
\end{block}\pause
\begin{itemize}
\item We will be seeing common security properties that happen on literature;\pause
\item The descriptions are solely the lecturer's opinion on the properties and may be wrong.
\end{itemize}
\end{frame}

\begin{frame}{List of Advanced Security Properties}
\begin{itemize}
\item Forward secrecy;\pause
\item Non-repudiation;\pause
\item Plausible Deniability;\pause
\item Availability;\pause
\item Eligibility;\pause
\item Fairness;\pause
\item Receipt-freeness;\pause
\item Coercion-resistance;\pause
\item Privacy; \pause
\item Anonymity;\pause
\item Transparency.
\end{itemize}
\end{frame}


\begin{frame}{Forward Secrecy}
\begin{itemize}
\item Forward secrecy relates to the non-interference of short term keys leakage to new short term keys;\pause
\item It does not relate to long term keys;\pause
\item Is usually obtained by the negotiation of short term keys only based on long term keys;\pause
\item Has a complementary property that is Backwards secrecy;\pause
\item Having both lead to full non-interference among session keys.
\end{itemize}
\end{frame}

\begin{frame}{Non-repudiation}
\begin{itemize}
\item Is the inability to deny knowledge of a message;\pause
\item Happens as non-repudiation of origin, meaning authorship;\pause
\item Happens as non-repudiation of destiny, meaning confirmation of reception;\pause
\item Is usually implemented using asymmetric crypto in digital signature mode;\pause
\item Can also be achieve by the use of commitments.
\end{itemize}
\end{frame}

\begin{frame}{Plausible Deniability}
\begin{itemize}
\item Is the ability to deny knowledge of a message;\pause
\item Act the the courter property of Non-Repudiation;\pause
\item Is implemented shared secret crypto;\pause
\item Can also happen on origin, destination or both.
\end{itemize}
\end{frame}

\begin{frame}{Availability}
\begin{itemize}
\item Is the property that relates to presence of knowledge whenever needed;\pause
\item Is difficult to reach by crypto-means;\pause
\item Is usually reached using replication;\pause
\item There are some interesting primitives that achieve availability such as secret-sharing.
\end{itemize}
\end{frame}

\begin{frame}{Eligibility}
\begin{itemize}
\item Is the property that states authority to a peer to act;\pause
\item Usually present on election protocols;\pause
\item Is related and derived from Authentication;\pause
\item Can also happen through delegation;\pause
\item Is implemented in this later case by the usage of tickets;\pause
\item Can also control the number of times the peer is allowed to do something.
\end{itemize}
\end{frame}

\begin{frame}{Fairness}
\begin{itemize}
\item Fairness is the properties that guarantees that no information is acquired out of the right time;\pause 
\item In election protocols it means that no early results can be obtained which could influence the remaining voters;\pause
\item Is usually implemented with encryption (either symmetric or asymmetric);\pause
\item The keys are them distributed is such a way that only an agreement can enable decryption.
\end{itemize}
\end{frame}

\begin{frame}{Receipt-freeness}
\begin{itemize}
\item Receipt-freeness is the property that the peer does not carry any proof of acts within the protocol;\pause
\item In election protocols it means that a voter does not gain any information (a receipt) which can be used to prove to a coercer that she voted in a certain way;\pause
\item It is tricky to achieve when combined with other properties;\pause
\item Implementation usually is not done using cryptographic means.
\end{itemize}
\end{frame}

\begin{frame}{Coercion-resistance}
\begin{itemize}
\item Coercion-resistance is the property that avoid a peer to act in certain way against its own will and forced by an external entity;\pause
\item In election protocols it means that a voter cannot cooperate with a coercer to prove to him that she voted in a certain way;\pause
\item It is usually achieve by using the last commitment within the protocol;\pause
\item Implementation usually depends of Receipt-freeness but is not a requirement.
\end{itemize}
\end{frame}

\begin{frame}{Verifiability}
\begin{itemize}
\item Is the property that allows for peers to be assured that their interaction was perceived within the protocol;\pause
\item Is usually implemented using bulletin boards;\pause
\item In election protocols it can be specialised in:\pause
\begin{itemize}
\item Individual verifiability: a voter can verify that her vote was really counted;\pause
\item Universal verifiability: the published outcome really is the sum of all the votes.
\end{itemize}
\end{itemize}
\end{frame}

\begin{frame}{Privacy}
\begin{itemize}
\item Privacy is the property that allows for peers to choose the amount of data that is being release to other peers;\pause
\item Has a controversial definition since it is related to a personal feeling;\pause
\item It is intrinsically related to Confidentiality;\pause
\item In election protocols it means that the system cannot reveal how a particular voter voted;\pause
\end{itemize}
\end{frame}

\begin{frame}{Anonymity}
\begin{itemize}
\item Anonymity is the property that does not allow identification of peers;\pause
\item Is usually achieve by using obfuscation techniques;\pause
\item Usually is implemented with hashes or MACs;\pause
\item Is a form of plausible deniability.
\end{itemize}
\end{frame}

\begin{frame}{Transparency}
\begin{itemize}
\item Transparency can be defined the distance of a message from ground truth;\pause
\item It a very new property that is actually being studied still;\pause
\item Is present in crypto-currency protocols and in health related systems;\pause
\item Is a dichotomy of Privacy.
\end{itemize}
\end{frame}

\begin{frame}{Discussion}
\begin{itemize}
\item Which other properties did you hear about?\pause
\item Which are the dichotomies you can see between the properties shown today?\pause
\item Can you foresee an online activity that you require a property not listed here?
\end{itemize}
\end{frame}


{
\usebackgroundtemplate{\includegraphics[width=\paperwidth,
height=\paperheight]{../reusable_images/fundo_capa.png}}
\begin{frame}

{\LARGE Questions????}

\end{frame}
}

{
\usebackgroundtemplate{\includegraphics[width=\paperwidth,
height=\paperheight]{../reusable_images/fundo_capa.png}}
\begin{frame}
\includegraphics[scale=0.8]{../reusable_images/cc_logo_arge.png}\hspace{0.5cm} 
\includegraphics[scale=0.95]{../reusable_images/by.png}

\vspace{1cm}
This work is licensed under the Creative Commons Attribution 4.0 International License. To view a copy of this license, visit http://creativecommons.org/licenses/by/4.0/.
\end{frame}
}

\end{document}