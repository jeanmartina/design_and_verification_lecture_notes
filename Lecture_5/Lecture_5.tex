\documentclass[12pt,table,xcolor={dvipsnames}]{beamer}
\usetheme{Pittsburgh}
\usecolortheme{seagull}
%\usepackage[utf8]{inputenc}
\usepackage{fontspec}
\usepackage{amsmath}
\usepackage{listings}
\usepackage{multirow}
\usepackage{amsfonts}
\usepackage{amssymb}
\usepackage{graphicx}
\author{Design and Verification of Security Protocols and Security Ceremonies}
\title{\vspace{-.2cm}Classical Protocols}
%\setbeamercovered{transparent} 
\setbeamertemplate{navigation symbols}{} 
%\logo{\includegraphics[scale=0.015]{Brasao_UFSC.png}\includegraphics[scale=0.2]{brasao_PPGCC.jpg}} 
\institute{Programa de Pós-Graduacão em Ciências da Computacão \\ Dr. Jean Everson Martina} 
\date{\vspace{.2cm}March-June 2018} 
\subject{} 
\usebackgroundtemplate{\includegraphics[width=\paperwidth,
height=\paperheight]{../reusable_images/fundo_UFSC.png}}
\begin{document}

{
\usebackgroundtemplate{\includegraphics[width=\paperwidth,
height=\paperheight]{../reusable_images/fundo_capa.png}}
\begin{frame}
\titlepage
\includegraphics[scale=0.3]{../reusable_images/brasao_PPGCC.jpg}
\end{frame}
}

\begin{frame}{Protocols to See Today!}
\begin{itemize}
\item Otway-Rees;\pause
\item Woo–Lam;\pause
\item Neuman–Stubblebine;\pause
\item Wide-Mouth Frog protocol;\pause
\item Yahalom.
\end{itemize}
\end{frame}

\begin{frame}{Otway-Rees Protocol}
\begin{itemize}
\item By Dave Otway and Owen Rees in 1987;\pause
\item A protocol for efficient mutual authentication (via a mutually trusted third party);\pause 
\item It assures both principal parties of the timeliness of the interaction without the use of clocks or double encipherment. 
\end{itemize}
\end{frame}


\begin{frame}{Otway-Rees Protocol}
\begin{table}[htdp]
\begin{center}
\begin{tabular}{ l l }
1. & $A\rightarrow B:M,A,B,\{N_{A},M,A,B\}_{K_{AS}}$ \\\pause
2. & $B\rightarrow S:M,A,B,\{N_{A},M,A,B\}_{{K_{{AS}}}},\{N_{B},M,A,B\}_{{K_{{BS}}}}$ \\\pause
3. & $S\rightarrow B:M,\{N_{A},K_{{AB}}\}_{{K_{{AS}}}},\{N_{B},K_{{AB}}\}_{{K_{{BS}}}}$ \\\pause
4. & $B\rightarrow A:M,\{N_{A},K_{{AB}}\}_{{K_{{AS}}}}$ 
\end{tabular}
\end{center}
\end{table}%
\end{frame}

\begin{frame}{Otway-Rees Protocol - Questions}
\begin{itemize}
\item What are the assumptions? \pause
\item What seems to be the goal?\pause 
\item What might the principals believe after each step?\pause
\item What is missing?\pause
\item Is it secure?
\end{itemize}
\end{frame}

\begin{frame}{Two Attacks on Otway-Rees Protocol}
\begin{itemize}
\item A malicious intruder can arrange for A and B to end up with
different keys:\pause
\begin{enumerate}
\item After step 3, B has received $K_{AB}$;\pause
\item An intruder then intercepts the fourth message;\pause
\item The intruder resends message 2, so S generates a new key $K'_{AB}$, sent to B;\pause
\item The intruder intercepts this message too, but sends to A the message $M,\{N_a,K_{AB}\}_{K_{AS}}$;\pause
\item A has $K'_{AB}$, while B has $K_{AB}$. \pause
\end{enumerate}
\item Another problem: although the server tells B that A used a nonce, B doesn’t know if this was a replay of an old message.
\end{itemize}
\end{frame}

\begin{frame}{Woo–Lam Protocol}
\begin{itemize}
\item Woo–Lam refers to various computer network authentication protocols designed by Simon S. Lam and Thomas Woo;\pause
\item The most common is the one published in 1992;\pause
\item The protocols enable two communicating parties to authenticate each other and to exchange session keys;\pause
\item It involves the use of a trusted key distribution center (KDC) to negotiate between the parties;\pause
\item Both symmetric-key and public-key variants have been described.
\end{itemize}
\end{frame}

\begin{frame}{Woo–Lam Protocol}
\begin{table}[htdp]
\begin{center}
\begin{tabular}{ l l }
1. & $A\rightarrow B: A$ \\\pause
2. & $B\rightarrow A: NB$ \\\pause
3. & $A\rightarrow B: \{NB\}_{K_{AS}} $ \\\pause
4. & $B\rightarrow S: \{A,\{NB\}_{K_{AS}}\}_{K_{BS}}$ \\\pause
5. & $S\rightarrow B: \{NB\}_{K_{BS}}$
\end{tabular}
\end{center}
\end{table}%
\end{frame}

\begin{frame}{Woo–Lam Protocol - Questions}
\begin{itemize}
\item What are the assumptions? \pause
\item What seems to be the goal?\pause 
\item What might the principals believe after each step?\pause
\item What is missing?\pause
\item Is it secure?
\end{itemize}
\end{frame}

\begin{frame}{Attack on Woo–Lam Protocol}
\begin{itemize}
\item Two simultaneous inbound authentication attempts initiated by an attacker C,
where C is also considered as any other regular participant; \pause
\item C pretends to be A in one and retains its own identity C for the other; \pause
\item C obtains nonces from B for both runs and encrypts the nonce NB intended for A with its own server key and returns it to B, retaining its original identity; \pause
\item When the nonce is returned by the server, it leads B to believe that it has authenticated A, whereas A has not even participated in either of the runs. 
\item The attack is complete. 
\end{itemize}
\end{frame}

\begin{frame}{Neuman–Stubblebine Protocol}
\begin{itemize}
\item Created by Neuman and Stubblebine in 1993;\pause
\item It allows individuals communicating over such a network to prove their identity to each other;\pause
\item This protocol utilizes time stamps, but does not depend on synchronized clocks;\pause
\item It has an Establishment phase and a Communication phase.
\end{itemize}
\end{frame}

\begin{frame}{Neuman–Stubblebine Protocol - Establishment}
\begin{table}[htdp]
\begin{center}
\begin{tabular}{ l l }
1. & $A\rightarrow B: A, N_{A}$ \\\pause
2. & $B\rightarrow S: B,N_{B},\{A,N_{A},T_{B}\}_{K_{BS}}$ \\\pause
3. & $S\rightarrow A:\{B,N_{A},K_{AB},T_{B}\}_{K_{AS}},\{A,K_{AB},T_{B}\}_{K_{BS}},N_{B}$ \\\pause
4. & $A\rightarrow B:\{A,K_{AB},T_{B}\}_{K_{BS}},\{N_{B}\}_{K_{AB}}$
\end{tabular}
\end{center}
\end{table}%
\end{frame}

\begin{frame}{Neuman–Stubblebine Protocol - Communication}
\begin{table}[htdp]
\begin{center}
\begin{tabular}{ l l }
1. & $A\rightarrow B: \{A,K_{AB},T_{B}\}_{K_{BS}},N'_{A}$ \\\pause
2. & $B\rightarrow A: N'_{B},\{N'_{A}\}_{K_{AB}}$ \\\pause
3. & $A\rightarrow B: \{N'_{B}\}_{K_{AB}}$ 
\end{tabular}
\end{center}
\end{table}%
\end{frame}


\begin{frame}{Neuman–Stubblebine Protocol - Questions}
\begin{itemize}
\item What are the assumptions? \pause
\item What seems to be the goal?\pause 
\item What might the principals believe after each step?\pause
\item What is missing?\pause
\item Is it secure?
\end{itemize}
\end{frame}

\begin{frame}{Attacks on Neuman–Stubblebine Protocol}
\begin{itemize}
\item Weidenbach Attack:
\item The server can be used by the attacker to generate an arbitrary number of messages $\{A,K_{AB},T_{B}\}_{K_{BS}}$. As the attacker knows that the only thing that changes is the key $K_{AB}$ he can make the Server to generate material for known-plain text attacks;
\end{itemize}
\end{frame}

\begin{frame}{Attacks on Neuman–Stubblebine Protocol}
\begin{itemize}
\item Paradox Attack:
\item While B sends message (2) to S, C intercepts the ciphertext {A, N x, Tb}Kbs and the nonce N b generated by B . C ignores the message (3) (bypasses Step (2) and Step (3)) and sends {A, N x, To}Kbs together with {Nb}N x as the message (4) to B. Because both
{A, Nx, Tb}Kbs and {A, Kab, Tb}Kbs have the same format  B cannot distinguish one from the other;
\end{itemize}
\end{frame}


\begin{frame}{Attacks on Neuman–Stubblebine Protocol}
\begin{itemize}
\item Oracle Attack:
\item C first intercepts the ticket in Establishment. Although C cannot decrypt, C may send the intercepted ticket and a forged nonce, $N'_{C}$, to B;\pause
\item Upon receiving the message, B verifies the ticket. If it is valid, B responds $\{N'_{C}\}_{K_{AB}}$ and a new nonce, $N'_{B}$, to A;\pause
\item Once C intercepts this message, he uses B as an oracle and starts a new session with
B;
\end{itemize}
\end{frame}

\begin{frame}{Attacks on Neuman–Stubblebine Protocol}
\begin{itemize}
\item C sends the nonce $N'_{B}$, he just received coupled the same ticket to B. Upon receiving the message, B sends back $\{N'_{C}\}_{K_{AB}}$ and a new nonce, $N''_{B}$, to A;\pause
\item X can intercept it and get the encrypted nonce $\{N'_{B}\}_{K_{AB}}$;\pause
\item Finally, C successfully passes the first authentication session of B by sending the $\{N'_{B}\}_{K_{AB}}$ back to B.
\end{itemize}
\end{frame}

\begin{frame}{Wide-Mouth Frog protocol Protocol}
\begin{itemize}
\item The protocol was first described under the name "The Wide-mouthed-frog Protocol" in the paper "A Logic of Authentication" (1990), which introduced BAN Logic;\pause
\item The paper gives no rationale for the protocol's whimsical name;\pause
\item It allows individuals communicating over a network to prove their identity to each other while also preventing eavesdropping or replay attacks, and provides for detection of modification and the prevention of unauthorized reading.
\end{itemize}
\end{frame}


\begin{frame}{Wide-Mouth Frog protocol Protocol}
\begin{table}[htdp]
\begin{center}
\begin{tabular}{ l l }
1. & $A\rightarrow S:A,\{T_{A},B,K_{{AB}}\}_{{K_{{AS}}}}$ \\\pause
2. & $S\rightarrow B:\{T_{S},A,K_{{AB}}\}_{{K_{{BS}}}}$ \\\pause
\end{tabular}
\end{center}
\end{table}%
To prevent active attacks, some form of authenticated encryption (or message authentication) must be used.
\end{frame}


\begin{frame}{Wide-Mouth Frog protocol - Questions}
\begin{itemize}
\item What are the assumptions? \pause
\item What seems to be the goal?\pause 
\item What might the principals believe after each step?\pause
\item What is missing?\pause
\item Is it secure?
\end{itemize}
\end{frame}

\begin{frame}{Problems on Wide-Mouth Frog protocol}
\begin{itemize}
\item A global clock is required;\pause
\item The server S has access to all keys;\pause
\item The value of the session key $K_{AB}$ is completely determined by A, who must be competent enough to generate good keys;\pause
\item It can replay messages within the period when the timestamp is valid. A is not assured that B exists;\pause
\item The protocol is stateful. This is usually undesired because it requires more functionality and capability from the server. For example, S must be able to deal with situations in which B is unavailable.
\end{itemize}
\end{frame}

\begin{frame}{Attack on Wide-Mouth Frog protocol}
\begin{itemize}
\item The value of the session key $K_{AB}$ is completely determined by an untrusted peer in the protocol;\pause
\item A Man-in-the-middle attack is trivial;\pause
\item C can deliberately reuse keys to defeat the protocols goals.
\end{itemize}
\end{frame}

\begin{frame}{Yahalom Protocol}
\begin{itemize}
\item Unpublished protocol. It appears on BAN Logic paper as personal communication by the authors with Yahalom;\pause
\item Yahalom uses a trusted arbitrator to distribute a shared key between two people;\pause
\item This protocol can be considered as an improved version of Wide Mouth Frog protocol.
\end{itemize}
\end{frame}


\begin{frame}{Yahalom Protocol}
\begin{table}[htdp]
\begin{center}
\begin{tabular}{ l l }
1. & $A\rightarrow B:A,N_{A}$ \\\pause
2. & $B\rightarrow S:B,\{A,N_{A},N_{B}\}_{K_{BS}}$ \\\pause
3. & $S\rightarrow A:\{B,K_{{AB}},N_{A},N_{B}\}_{{K_{{AS}}}},\{A,K_{{AB}}\}_{{K_{{BS}}}}$ \\\pause
4. & $A\rightarrow B:\{A,K_{{AB}}\}_{{K_{{BS}}}},\{N_{B}\}_{{K_{{AB}}}}$ 
\end{tabular}
\end{center}
\end{table}%
\end{frame}

\begin{frame}{Yahalom Protocol - Questions}
\begin{itemize}
\item What are the assumptions? \pause
\item What seems to be the goal?\pause 
\item What might the principals believe after each step?\pause
\item What is missing?\pause
\item Is it secure?
\end{itemize}
\end{frame}

\begin{frame}{Attacks on Yahalom Protocol}
\begin{itemize}
\item Bob completed his protocol execution believing he was communicating with Alice, but it actually was not so;\pause
\item Because the first encrypted chunk in the fourth message does not include the terms used for proving the freshness of the session key, such as NB, the encrypted chunk
could be a replayed message. 
\end{itemize}
\end{frame}

\begin{frame}{Discussion}
\begin{itemize}
\item What the classical protocols tell us in terms of designs?\pause
\item What the classical protocols tell us in terms of flaws?\pause
\item How most of these flaws were discovered?
\end{itemize}
\end{frame}


{
\usebackgroundtemplate{\includegraphics[width=\paperwidth,
height=\paperheight]{../reusable_images/fundo_capa.png}}
\begin{frame}

{\LARGE Questions????}

\end{frame}
}

{
\usebackgroundtemplate{\includegraphics[width=\paperwidth,
height=\paperheight]{../reusable_images/fundo_capa.png}}
\begin{frame}
\includegraphics[scale=0.8]{../reusable_images/cc_logo_arge.png}\hspace{0.5cm} 
\includegraphics[scale=0.95]{../reusable_images/by.png}

\vspace{1cm}
This work is licensed under the Creative Commons Attribution 4.0 International License. To view a copy of this license, visit http://creativecommons.org/licenses/by/4.0/.
\end{frame}
}

\end{document}